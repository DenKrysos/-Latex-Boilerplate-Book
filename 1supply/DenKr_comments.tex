%==================================================================================
% ----  Some Stuff to help during working state
% -- Enables you this infamous comments that helps gratuitously while concurrent working / reviewing among several Co-Authors.
% -- Adjust Author-Names and Macro-Names to your Project
% % % % % % % % % % % % % % % % % % % % %
% -- USAGE - Extent by additional Macros
% -- -- For adding more Persons beyond the current number
% - Just duplicate one of these blocks below and adjust to your requirements (Name of the Cmd and Name-Tag (Abbreviation for Persons Name) inside.
% - The "\newcommandsDisw{#1}{#2}{#3}" creates two Macros for use.
%    - Arguments:
%        1: Name for the created Macros
%        2: Abbreviation of the Person's Name to display
%         [Optional]: It has an optional Argument after the second (\newcommandsDisw{#1}{#2}[#Opt1]{#3})
%                   -> If this is supplied, its content is printed in the Legend. Otherwise, #2 is put to the Legend.
%        3: The Color to use
%    - Created Macros:
%        "Tagged Comment" -> \<Abbr>{}: Prints text, suffixed with "#<Abbr>: " (and colored)
%        "Highlighting" -> \<Abbr>hl{}: Just prints the text in Color.
%    -> E.g.: \newcommandsDisw{dekr}{DenKr}{<Color>} creates the two Macros
%        \dekr{}
%        \dekrhl{}
% - Colors are defined up to "denkrComCol12"
% - Above that, the only thing missing: Define a new Color. Just add a Color-Definition according to xcolor.sty, e.g.
%        \definecolor{denkrComCol<Something>}{named}{blue}%
%   above the duplicated Macro-Definition and then use that Color's name (what you've written as "denkrComCol<Something>")
%
% - You may, at some point in time, overwrite single commands (like for example shown below with the \dekr{} ones).
%   - For example, during writing, you use a highlight Command to, ehem, highlight text in your color.
%      Approaching release, you maybe don't want to just deactivate and thus remove this text, but simply remove the color-highlighting.
%      Then you could redefine the \<abbr>hl{} cmd to a simply "pasting the argument".
%----------------------------------------------------------------------------------
%
%= = = = = = = = = = = = = = = = = = = = = = = = = = = = = = = = = = = = = = = = = =
% - - - - - - - - - - - - - - - - - - - - - - - - - - - - - - - - - - - - - - - - -
\newcommandsDisw{dekr}{DenKr}[Dennis Krummacker]{denkrComColDenKr}%
% \renewcommand{\dekr}[1]{}%% Disable comments [selectively this]
% \renewcommand{\dekrhl}[1]{#1}%% Disable highlighting of text [selectively this]
% \renewcommand{\dekrhl}[1]{}%% Completely disable this kind of text [selectively this]
%
\newcommandsDisw{chfi}{ChFi}[Christoph Fischer]{denkrComCol1}%
%
\newcommandsDisw{beve}{BeVe}[Benedikt Veith]{denkrComCol2}%
%
\newcommandsDisw{frpo}{FrPo}[Franc Pouhela]{denkrComCol3}%
%
\newcommandsDisw{dali}{DaLi}[Daniel Lindenschmitt]{denkrComCol4}%
%
\newcommandsDisw{desa}{DeSa}[Dennis Salzmann]{denkrComCol5}%
%
\newcommandsDisw{hedi}{Hedi}[Mohamed Romdhane]{denkrComCol6}%
%
\newcommandsDisw{dummyA}{DummyA}[Just-some-Dummy]{denkrComCol7}%
%
\newcommandsDisw{dummyB}{DummyB}[Just-some-Dummy]{denkrComCol8}%
%
\newcommandsDisw{dummyC}{DummyC}[Just-some-Dummy]{denkrComCol9}%
%
\newcommandsDisw{dummyD}{DummyD}[Just-some-Dummy]{denkrComCol10}%
%
% - - - - - - - - - - - - - - - - - - - - - - - - - - - - - - - - - - - - - - - - -
%= = = = = = = = = = = = = = = = = = = = = = = = = = = = = = = = = = = = = = = = = =