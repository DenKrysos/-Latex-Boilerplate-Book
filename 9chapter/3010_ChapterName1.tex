
\chapter{Chapter 1}
\label{chap:chapter1}

Example-Citation:
\cite{DenKr_denkrement1_indeco}
\nl%
Some special characters:
»«.
\nl%
Some Example Macro:
\gerguiquote{My Enquote}.
\nl%
Example Acronym:
\gls{fsa}


\section{Testing Stuff}
\label{sec:testing}

Document-within References to check whether hyperref \& nameref work properly:\nl%
\ref{chap:intro} (\nameref{chap:intro}). \ref{chap:chapter1}  (\nameref{chap:chapter1}). \ref{sec:testing} (\nameref{sec:testing})\nl%
Consider using my Macros:\nl
\eleref{fig:tikz_test2} or \elenumnamref{fig:tikzexample}

\npi%
\begin{itemize}
\item%
    Example Itemization
\end{itemize}
\begin{enumerate}
\item%
    Example Enumeration
\end{enumerate}
\begin{description}
\item[Description Item Label]%
    An Example Description List, with a longer Parapgraph. Just some random Text without any sense, but with sufficient length to cause a line-break, so that the indentation is properly showing influence.
\end{description}
\begin{enuminlrom}
\item%
    Example Inline Roman Enumeration
\item%
    Item 2.
\end{enuminlrom}





\section{\textbackslash gls{} in Heading: \glstextHead{qos}, \glsdescHead{qos}, \glstextplHead{qos}, \glsdescplHead{qos}}
These macros allow to use glossaries entries in Section-Heading, without the Warning \enquote{Token not allowed}.





\section{Japanese Font -- 日本語}

Ordinary | \textrm{Serif} | \textsf{Sans-Serif} | \texttt{Monospaced-TrueType} | \textbf{Bold} | \textit{Italic} | \textit{\textbf{BoldItalic}} | \textsc{SmallCaps}\nl%
Some Japanese Characters:\nl%
\begin{tabular}{llll}%
Ordinary:&日本語行無&いきます&チョット\\%
Bold:&\textbf{日本語行無}&\textbf{いきます}&\textbf{チョット}\\%
Serif (Mincho):&\textrm{日本語行無}&\textrm{いきます}&\textrm{チョット}\\%
Sans-Serif (Gothic):&\textsf{日本語行無}&\textsf{いきます}&\textsf{チョット}\\%
Bold-Sans:&\textbf{\textsf{日本語行無}}&\textbf{\textsf{いきます}}&\textbf{\textsf{チョット}}\\%
Monospaced:&\texttt{日本語行無}&\texttt{いきます}&\texttt{チョット}\\%
Mono-Bold:&\textbf{\texttt{日本語行無}}&\textbf{\texttt{いきます}}&\textbf{\texttt{チョット}}\\%
\hline%
Italic:&\textit{日本語行無}&\textit{いきます}&\textit{チョット}\\%
Serif-It:&\textit{\textrm{日本語行無}}&\textit{\textrm{いきます}}&\textit{\textrm{チョット}}\\%
Sans-It:&\textit{\textsf{日本語行無}}&\textit{\textsf{いきます}}&\textit{\textsf{チョット}}\\%
Mono-It:&\textit{\texttt{日本語行無}}&\textit{\texttt{いきます}}&\textit{\texttt{チョット}}\\%
\hline%
SmallCaps:&\textsc{日本語行無}&\textsc{いきます}&\textsc{チョット}\\%
% Serif-SC:&\textsc{\textrm{日本語行無}}&\textsc{\textrm{いきます}}&\textsc{\textrm{チョット}}\\%
% Sans-SC:&\textsc{\textsf{日本語行無}}&\textsc{\textsf{いきます}}&\textsc{\textsf{チョット}}\\%
% Mono-SC:&\textsc{\texttt{日本語行無}}&\textsc{\texttt{いきます}}&\textsc{\texttt{チョット}}\\%
\end{tabular}%


\np
Imported Standalone Snippet: -->
\import{\DenKrContentRootDir/}{3020_StandaloneConstituent_Snippet}
<--
% \np
% As Standalone-Inclusion: -->
% \includestandalone[mode=\includestandalonedefaultmode]{\DenKrContentRootDir/3020_StandaloneConstituent_Snippet}%
% <--
% \np


\npi%
{%
    \Huge%
    \definecolor{CharFillColor}{rgb}{1,.8,.8}%
    \colorlet{CharFillColor}{orange}%
    \sffamily%
    \textcolor{violet}{%
    \textpdfrender{%
        TextRenderingMode=FillStroke,
        LineWidth=\DenKrOutlineWidth,
        FillColor=CharFillColor,
        MiterLimit=1,
    }{Text with Outline/Contour}}\nl%
    \textcolor{violet}{%
    \textpdfrender{%
        TextRenderingMode=FillStroke,
        LineWidth=0.06em,
        FillColor=CharFillColor,
        MiterLimit=1,
    }{Text with Outline/Contour}}%
}%


\np
%
\begin{figure}[!htpb]
    \centering
	\includestandalone[mode=\includestandalonedefaultmode]{\tikzFilesPath/tikztest}% width=\columnwidth
    \caption{Tikz-Picture Caption. Some Example Standalone-Tikz-Pic}
    \label{fig:tikz_test}
\end{figure}
%
%
\tikzabb%[tex]%
{tikztest}%
[\DenKrLayoutMainRootDir/8templates/tikz/7Tikz]% Alternative Path to the std "./7Tikz"
{%
Tikz-Picture Caption. Some Example Standalone-Tikz-Pic
}%
{1}[1,1,!ht]%[0.55,1,!ht]%
{fig:tikz_test2}%
%





\clearpage
Make this Chapter end on a right page (odd number), to check whether following Chapter starts right.