%  A Template for the File "./1supply/DenKr_comments.tex",
%    which concretely configures the __sty.tex for using the Comment-Function
%  In case not present, create the File "./1supply/DenKr_comments.tex", copy the stuff here below this Comment over and adjust to your needs.
%
%
%==================================================================================
% ----  Some Stuff to help during working state
% -- Enables you this infamous comments that helps gratuitously while concurrent working / reviewing among several Co-Authors.
% -- Adjust Author-Names and Macro-Names to your Project
% % % % % % % % % % % % % % % % % % % % %
% -- USAGE - Extent by additional Macros
% -- -- For adding more Persons beyond the current number
% - Just duplicate one of these blocks below and adjust to your requirements (Name of the Cmd and Name-Tag (Abbreviation for Persons Name) inside.
% - The "\newcommandsDisw{#1}{#2}{#3}" creates two Macros for use.
%    - Arguments:
%        1: Name for the created Macros
%        2: Abbreviation of the Person's Name to display
%         [Optional]: It has an optional Argument after the second (\newcommandsDisw{#1}{#2}[#Opt1]{#3})
%                   -> If this is supplied, its content is printed in the Legend. Otherwise, #2 is put to the Legend.
%        3: The Color to use
%    - Created Macros:
%        "Tagged Comment" -> \<Abbr>{}: Prints text, suffixed with "#<Abbr>: " (and colored)
%        "Highlighting" -> \<Abbr>hl{}: Just prints the text in Color.
%    -> E.g.: \newcommandsDisw{dekr}{DenKr}{<Color>} creates the two Macros
%        \dekr{}
%        \dekrhl{}
% - Colors are defined up to "denkrComCol12"
% - Above that, the only thing missing: Define a new Color. Just add a Color-Definition according to xcolor.sty, e.g.
%        \definecolor{denkrComCol<Something>}{named}{blue}%
%   above the duplicated Macro-Definition and then use that Color's name (what you've written as "denkrComCol<Something>")
%
% - You may, at some point in time, overwrite single commands (like for example shown below with the \dekr{} ones).
%   - For example, during writing, you use a highlight Command to, ehem, highlight text in your color.
%      Approaching release, you maybe don't want to just deactivate and thus remove this text, but simply remove the color-highlighting.
%      Then you could redefine the \<abbr>hl{} cmd to a simply "pasting the argument".
%----------------------------------------------------------------------------------
%
%= = = = = = = = = = = = = = = = = = = = = = = = = = = = = = = = = = = = = = = = = =
% - - - - - - - - - - - - - - - - - - - - - - - - - - - - - - - - - - - - - - - - -
\newcommandsDisw{dekr}{DenKr}[Dennis Krummacker]{denkrComColDenKr}%
% \renewcommand{\dekr}[1]{}%% Disable comments [selectively this]
% \renewcommand{\dekrhl}[1]{#1}%% Disable highlighting of text [selectively this]
% \renewcommand{\dekrhl}[1]{}%% Completely disable this kind of text [selectively this]
%
\newcommandsDisw{chfi}{ChFi}[Christoph Fischer]{denkrComCol1}%
%
\newcommandsDisw{beve}{BeVe}[Benedikt Veith]{denkrComCol2}%
%
\newcommandsDisw{frpo}{FrPo}[Franc Pouhela]{denkrComCol3}%
%
\newcommandsDisw{dali}{DaLi}[Daniel Lindenschmitt]{denkrComCol4}%
%
\newcommandsDisw{desa}{DeSa}[Dennis Salzmann]{denkrComCol5}%
%
\newcommandsDisw{hedi}{Hedi}[Mohamed Romdhane]{denkrComCol6}%
%
\newcommandsDisw{dummyA}{DummyA}[Just-some-Dummy]{denkrComCol7}%
%
\newcommandsDisw{dummyB}{DummyB}[Just-some-Dummy]{denkrComCol8}%
%
\newcommandsDisw{dummyC}{DummyC}[Just-some-Dummy]{denkrComCol9}%
%
\newcommandsDisw{dummyD}{DummyD}[Just-some-Dummy]{denkrComCol10}%
%
% - - - - - - - - - - - - - - - - - - - - - - - - - - - - - - - - - - - - - - - - -
%= = = = = = = = = = = = = = = = = = = = = = = = = = = = = = = = = = = = = = = = = =
%
%
%
%
%= = = = = = = = = = = = = = = = = = = = = = = = = = = = = = = = = = = = = = = = = =
% - - - - - - - - - - - - - - - - - - - - - - - - - - - - - - - - - - - - - - - - -
% Historical Explanation:
%  Either define such a "\TempDisplayPreparation" like here manually and paste this in the "postamble" (or anywhere, but suggested towards the document's end);
%  Or do nothing by yourself but simply paste "\DisplayDenKrCommandsLegend" (which is the current set-up in my Boilerplate).
%
% - I introduced the "automatic" Version (the \DisplayDenKrCommandsLegend) somewhen later (hence the "historical")
% - And basically, as this auto-version exists, there shouldn't be any need anymore to prepare a manual Legend, as this is populated automatically by the Command-Definitions and then rolls the displaying automatically out.
%
%\newcommand{\TempDisplayPreparation}{\disablewr{%
%	{%
%	\section{Draft-State: Comment Color Code}{\noindent\raggedright%
%	\todo{Comments: ToDos: \mbox{$\Rightarrow$ \textbackslash todo\{ \}}}\nl%
%	\notice{Comment: A general Notice: \mbox{$\Rightarrow$ \textbackslash notice\{ \}}}\nl%
%	\pasteColCodeLegend{dekr}{Dennis Krummacker}%
%	\pasteColCodeLegend{chfi}{Christoph Fischer}%
%	\pasteColCodeLegend{mabe}{Maximilian Berndt}%
%	\pasteColCodeLegend{jaze}{Janis Zemitis}%
%	\pasteColCodeLegend{beve}{Benedikt Veith}%
%	\pasteColCodeLegend{hedi}{Mohamed Romdhane}%
%	\pasteColCodeLegend{frpo}{Franc Pouhela}%
%	\pasteColCodeLegend{dummyA}{Just-some-Dummy}%
%	\pasteColCodeLegend{dummyB}{Just-some-Dummy}%
%	\pasteColCodeLegend{dummyC}{Just-some-Dummy}%
%	\pasteColCodeLegend{dummyD}{Just-some-Dummy}%
%	\textbf{$\Rightarrow$ You may add such a Comment-Macro for yourself in "\DenKrSupplyRootDir/DenKr\_comments.tex" if you wish $\Leftarrow$}%
%	}}%
%}}%
%----------------------------------------------------------------------------------
% - - - - - - - - - - - - - - - - - - - - - - - - - - - - - - - - - - - - - - - - -
%==================================================================================