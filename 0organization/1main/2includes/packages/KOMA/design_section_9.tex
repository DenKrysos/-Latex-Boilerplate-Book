\setkomafont{section}{\sffamily\bfseries\Large\color{DenKrKomaColor_SectionHeading}}%
\RedeclareSectionCommand[%
    beforeskip=5.5ex,%
    afterskip=2ex,%
]{section}%
\renewcommand*{\sectionformat}{\usekomafont{section}\thesection}%
%
\colorlet{secTFrame}{DenKrKomaColor_ChapterHeading!60!white}%
\colorlet{secTBG}{DenKrKomaColor_ChapterHeading_2!60!white}%
\colorlet{secTBGDark}{secTBG!60!black}%
\colorlet{secTBGFakeEdge}{black!80!white}%
%
\let\sectionMinHeight\relax%
\newlength{\sectionMinHeight}%
\settoheight{\sectionMinHeight}{\usekomafont{section}ÜÉÂgqQ}%
\setlength{\sectionMinHeight}{1.4\sectionMinHeight}%
%
\makeatletter%
\newif\if@DeKrRight%
\renewcommand*{\DenKrKOMAHookFormatSection}[4]{%
    \def\bandIndentOut{0.7cm}%
    \def\bandIndentIn{0.2cm}%
    \begin{tikzpicture}[remember picture]%
        \if@twoside% Twoside Document
            \Ifthispageodd{%Right
                \@DeKrRighttrue
            }{%Left
                \@DeKrRightfalse
            }%
        \else% Oneside Document: Treat every Page as 'right'
            \@DeKrRighttrue
        \fi%
        %
        \if@DeKrRight%Right
            \def\secTitleIndent{\bandIndentIn}%
        \else%
            \def\secTitleIndent{0pt}%
        \fi%
        \begin{pgfonlayer}{foreground}%
            \coordinate(section_anchor)at(0,0);%
            \node[draw=none,line width=0pt,shape=rectangle,anchor=north west,inner sep=0pt,minimum height=\sectionMinHeight](section_title)at($(section_anchor)+(\secTitleIndent,0)$){%
                \parbox[b]{\dimexpr\textwidth-\bandIndentIn\relax}{%
                    #3
                    \scalebox{0.8}{\textcolor{DenKrKomaColor_ChapterHeading}{\DenKrKomaHeadingBarockCrossAChar}}
                    #4%
                }%
            };%
        \end{pgfonlayer}%
        % Set some values/sizes
        \path let\n1={0.8ex},\n2={5em}in%
            coordinate(bgSepOut)at(\bandIndentOut,\n1)%
            coordinate(bgSepIn)at(\bandIndentIn,\n1)%
            coordinate(bgFadeLen)at(\n2,0.5*\n2)%X: Out, Y: In
            ;%
        % Update bounding-box and thus make the shift of the title node take effect. (Adding also the Y-component creates the appropriate size downwards)
        \path[draw=none,line width=0pt]let\p1=(bgSepOut)in(section_anchor)--($(section_title.south west)-(0,0.5*\y1+0.1ex)$)--($(section_title.north west)+(0,1.3ex)$);%
        % "Bake" the previous typeset pieces and "decouple" a new bounding box.
        \useasboundingbox%
            (current bounding box.north west)%
            rectangle%
            (current bounding box.south east);%
        \begin{pgfonlayer}{background}%
        \if@DeKrRight%Right
            \path let\p1=(bgSepOut),\p2=(bgSepIn),\p3=(bgFadeLen)in%
                coordinate(bgLT_T)at($(section_title.north west)+(0,\y2)$)%
                coordinate(bgLT_L)at($(section_title.north west)+(-\x2,0)$)%
                coordinate(bgLB_B)at($(section_title.south west)+(0,-\y2)$)%
                coordinate(bgLB_L)at($(section_title.south west)+(-\x2,0)$)%
                coordinate(bgRT)at($(section_title.north east)+(\x1,\y1)$)%
                coordinate(bgRB)at($(section_title.south east)+(\x1,-\y1)$)%
                coordinate(bgFadeLT)at($(bgLT_T)+(\y3,0)$)%
                coordinate(bgFadeLB)at($(bgLB_B)+(\y3,0)$)%
                coordinate(bgFadeRT)at($(bgRT)+(-\x3,0)$)%
                coordinate(bgFadeRB)at($(bgRB)+(-\x3,0)$)%
                coordinate(lappetR_T)at($(bgRB)-(\bandIndentOut-0.8ex,0)$)%
                coordinate(lappetR_B)at($(lappetR_T)+(0,-2ex)$)%
                coordinate(lappetL_B)at($(bgLT_T)+(3ex,0)$)%
                coordinate(lappetL_T)at($(lappetL_B)+(-0.2ex,1.3ex)$)%
                ;%
            %
            \path[draw=secTFrame,fill=secTBGDark,line width=0.1ex,line join=round](bgRB)to[out=260,in=70](lappetR_B)--(lappetR_T)--cycle;%
            \path[draw=secTBGFakeEdge,line width=0.15ex](lappetR_T)--(lappetR_B)--++(0,-0.3ex)coordinate(lappetR_Bfadeout);%
            %Why this first unconnected node here: Doesn't do anything for the path. But without it, some inverse-matrix fancy shenanigan shit fails to dimension too large. The coordinate extends the bounding box to the side, nothing else.
            \path[draw=secTBGFakeEdge,line width=0.15ex,path fading=south]($(lappetR_Bfadeout)+(-0.1ex,0)$)(lappetR_Bfadeout)--++(0,-1.2ex);%
            \path[shade,left color=white,right color=secTBG](bgRT)--(bgRB)--(bgFadeRB)--(bgFadeRT)--cycle;%
            \path[shade,left color=white,right color=secTFrame,line cap=round,line join=round]%
                let\n1={0.2ex}in%
                ($(bgRT)+(0,0.1ex)$)--(current subpath start-|bgFadeRT)--++(0,-\n1)-|cycle%
                ($(bgRB)+(0,-0.1ex)$)--(current subpath start-|bgFadeRB)--++(0,\n1)-|cycle%
            ;%
            \path[draw=secTFrame,line width=0.2ex,line cap=round,line join=round]%
                (bgRT)--(bgRB);%
            %
            %
            \path(bgLT_T)to[out=180,in=90]coordinate[pos=0.5](lappetL_S)(bgLT_L);%
            \path[draw=secTFrame,fill=secTBGDark,line width=0.1ex,line join=round](lappetL_S)to[out=40,in=220](lappetL_T)to[out=290,in=90](lappetL_B)--cycle;%
            \path[shade,left color=secTBG,right color=white]%
                (bgLT_T)to[out=180,in=90](bgLT_L)--(bgLB_L)to[in=180,out=270](bgLB_B)%
                --(bgFadeLB)--(bgFadeLT)--cycle;%
            \path[shade,left color=secTFrame,right color=white,line cap=round,line join=round]%
                let\n1={0.2ex}in%
                ($(bgLT_T)+(0,0.1ex)$)--(current subpath start-|bgFadeLT)--++(0,-\n1)-|cycle%
                ($(bgLB_B)+(0,-0.1ex)$)--(current subpath start-|bgFadeLB)--++(0,\n1)-|cycle%
            ;%
            \path[draw=secTFrame,line width=0.2ex,line cap=round,line join=round]%
                (bgLT_T)to[out=180,in=90](bgLT_L)--(bgLB_L)to[in=180,out=270](bgLB_B)%
                ;%
        \else%Left
            \path let\p1=(bgSepOut),\p2=(bgSepIn),\p3=(bgFadeLen)in%
                coordinate(bgLT)at($(section_title.north west)+(-\x1,\y1)$)%
                coordinate(bgLB)at($(section_title.south west)+(-\x1,-\y1)$)%
                coordinate(bgRT_T)at($(section_title.north east)+(0,\y2)$)%
                coordinate(bgRT_R)at($(section_title.north east)+(\x2,0)$)%
                coordinate(bgRB_B)at($(section_title.south east)+(0,-\y2)$)%
                coordinate(bgRB_R)at($(section_title.south east)+(\x2,0)$)%
                coordinate(bgFadeLT)at($(bgLT)+(\x3,0)$)%
                coordinate(bgFadeLB)at($(bgLB)+(\x3,0)$)%
                coordinate(bgFadeRT)at($(bgRT_T)+(-\y3,0)$)%
                coordinate(bgFadeRB)at($(bgRB_B)+(-\y3,0)$)%
                coordinate(lappetL_T)at($(bgLB)+(\bandIndentOut-0.8ex,0)$)%
                coordinate(lappetL_B)at($(lappetL_T)+(0,-2ex)$)%
                coordinate(lappetR_B)at($(bgRT_T)+(-3ex,0)$)%
                coordinate(lappetR_T)at($(lappetR_B)+(0.2ex,1.3ex)$)%
                ;%
            %
            \path[draw=secTFrame,fill=secTBGDark,line width=0.1ex,line join=round](bgLB)to[out=280,in=110](lappetL_B)--(lappetL_T)--cycle;%
            \path[draw=secTBGFakeEdge,line width=0.15ex](lappetL_T)--(lappetL_B)--++(0,-0.3ex)coordinate(lappetL_Bfadeout);%
            %This first unconnected Node: See above
            \path[draw=secTBGFakeEdge,line width=0.15ex,path fading=south]($(lappetL_Bfadeout)+(-0.1ex,0)$)(lappetL_Bfadeout)--++(0,-1.2ex);%
            \path[shade,left color=secTBG,right color=white](bgLT)--(bgLB)--(bgFadeLB)--(bgFadeLT)--cycle;%
            \path[shade,left color=secTFrame,right color=white,line cap=round,line join=round]%
                let\n1={0.2ex}in%
                ($(bgLT)+(0,0.1ex)$)--(current subpath start-|bgFadeLT)--++(0,-\n1)-|cycle%
                ($(bgLB)+(0,-0.1ex)$)--(current subpath start-|bgFadeLB)--++(0,\n1)-|cycle%
            ;%
            \path[draw=secTFrame,line width=0.2ex,line cap=round,line join=round]%
                (bgLT)--(bgLB);%
            %
            %
            \path(bgRT_T)to[out=0,in=90]coordinate[pos=0.5](lappetR_S)(bgRT_R);%
            \path[draw=secTFrame,fill=secTBGDark,line width=0.1ex,line join=round](lappetR_S)to[out=140,in=-40](lappetR_T)to[out=250,in=90](lappetR_B)--cycle;%
            \path[shade,left color=white,right color=secTBG]%
                (bgRT_T)to[out=0,in=90](bgRT_R)--(bgRB_R)to[in=0,out=270](bgRB_B)%
                --(bgFadeRB)--(bgFadeRT)--cycle%
                ;%
            \path[shade,left color=white,right color=secTFrame,line cap=round,line join=round]%
                let\n1={0.2ex}in%
                ($(bgRT_T)+(0,0.1ex)$)--(current subpath start-|bgFadeRT)--++(0,-\n1)-|cycle%
                ($(bgRB_B)+(0,-0.1ex)$)--(current subpath start-|bgFadeRB)--++(0,\n1)-|cycle%
            ;%
            \path[draw=secTFrame,line width=0.2ex,line cap=round,line join=round]%
            (bgRT_T)to[out=0,in=90](bgRT_R)--(bgRB_R)to[in=0,out=270](bgRB_B)%
                ;%
        \fi%
    \end{pgfonlayer}%
    \end{tikzpicture}%
    % \par%
}%
\makeatother%