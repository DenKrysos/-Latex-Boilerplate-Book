%________________________________________________________________________
%------------------------------------------------------------------------
%							Dokument Formatierungen
%/\/\/\/\/\/\/\/\/\/\/\/\/\/\/\/\/\/\/\/\/\/\/\/\/\/\/\/\/\/\/\/\/\/\/\/\
%---------------------------------------------------------
%					Non-KOMA
%---------------------------------------------------------
% Set Page-Numbering to Capital-Roman in the Front-Matter, to Arabic in the Main-Matter and prevent both from resetting the Page-Number-Counter
%       % Remark that \pagenumbering resets the Page Counter. Hence, not done like:
%        % \renewcommand{\frontmatter}{\cleardoublepage\@mainmatterfalse\pagenumbering{Roman}}
%        % \renewcommand{\mainmatter}{\cleardoublepage\@mainmattertrue\pagenumbering{arabic}}
\makeatletter%
\renewcommand{\frontmatter}{\cleardoublepage\@mainmatterfalse\renewcommand{\thepage}{\Roman{page}}}%
\renewcommand{\mainmatter}{\cleardoublepage\@mainmattertrue\renewcommand{\thepage}{\arabic{page}}}%
\makeatother%
%
% - - - - - - - -
% Block-Stretching across page
% - - - - - - - -
%     LaTeX uses \flushbottom for two-sided documents (book by default). Odd pages and even pages are forced to be aligned. In one-sided documents (article, report by default) LaTeX uses \raggedbottom, extra spaces will be gone. cf. classes document.
%     One can use \raggedbottom if met too many bad page breaks. However, it is preferred to prevent big boxes in your document. Use floats instead of put big tabulars and figures directly. For lists and section titles, it is often not too serious, be sure you put enough text for each sections.
\raggedbottom%
%
% - - - - - - - -
% Reset the Chapter-Counter with every Part
%  (When doing this, make sure to turn the Chapter-ID (\thechapter) into something Part-specific. Similar to how it is done with Sections and below.)
% With issuing the cmd, the document is changed from this spot on. Just like \mainmatter and alike
% - - - - - - - -
% USAGE:
%  When intending to use this, also make sufficient Space in the printing of the Table-Of-Contents. For that just utilize the provided Macro below.
% -> Remove the comment-percent in Front of the pasted Macro right at the bottom of the definition
% - - - - - - - -
\makeatletter%
\newcounter{matterPartSeparationStatus}%
\setcounter{matterPartSeparationStatus}{0}%
\newcounter{PartSeparationChapterBkp}%
\newcommand{\matterPartSeparatedActivate}{%
    \ifnumcomp{\value{matterPartSeparationStatus}}{>}{0}{}{%
        \setcounter{matterPartSeparationStatus}{1}%
        \setcounter{PartSeparationChapterBkp}{\value{chapter}}%
        \@addtoreset{chapter}{part}%
        \renewcommand{\thechapter}{\Alph{part}\arabic{chapter}}%
    }%
}%
\newcommand{\matterPartSeparatedRestore}{%
    \ifnumcomp{\value{matterPartSeparationStatus}}{>}{0}{%
        \@removefromreset{chapter}{part}%
        \setcounter{chapter}{\value{PartSeparationChapterBkp}}%
        \renewcommand{\thechapter}{\arabic{chapter}}%
        \setcounter{matterPartSeparationStatus}{0}%
    }{}%
}%
\makeatother%
\newcommand{\addSpaceToTocNumForPartSeparationMatter}{%
    \AddToHook{begindocument/before}{%
        \newlength{\partnumTOClength}%
        \setlength{\partnumTOClength}{1.5em}%
        \addtolength{\cftpartnumwidth}{\partnumTOClength}%
        \addtolength{\cftchapnumwidth}{\partnumTOClength}%
        \addtolength{\cftsecnumwidth}{\partnumTOClength}%
        \addtolength{\cftsubsecnumwidth}{\partnumTOClength}%
        \addtolength{\cftsubsubsecnumwidth}{\partnumTOClength}%
        \addtolength{\cftparanumwidth}{\partnumTOClength}%
        \addtolength{\cftsubparanumwidth}{\partnumTOClength}%
        \addtolength{\cftfignumwidth}{\partnumTOClength}%
        %\addtolength{\cftsubfignumwidth}{\partnumTOClength}%
        \addtolength{\cfttabnumwidth}{\partnumTOClength}%
        %\addtolength{\cftsubtabnumwidth}{\partnumTOClength}%
        \let\partnumTOClength\undefined%
    }%
}%
% \addSpaceToTocNumForPartSeparationMatter%
%
%
%
%---------------------------------------------------------
%					KOMA
%---------------------------------------------------------
\KOMAoptions{%
% 	BCOR=8mm,% Bindeverlust von 8mm am Innenrand einbeziehen
	DIV=last,%
}%
%
%________________________________________________________________________
% Colors
% - - (Define Colors collected on one spot, because these can be "intercepted")
%  - - (Defined them in the included File and instead these Colors are used)
%========================================================================
\definecolor{DenKrKomaColor_HeadLine}{RGB}{40,60,150}%
\definecolor{DenKrKomaColor_FootLine}{named}{DenKrKomaColor_HeadLine}%
\definecolor{DenKrKomaColor_PageNumber}{RGB}{0,60,150}%
%
\definecolor{DenKrKomaColor_PartHeading}{RGB}{40,0,100}%
\definecolor{DenKrKomaColor_ChapterHeading}{RGB}{60,0,150}%
\definecolor{DenKrKomaColor_SectionHeading}{named}{DenKrKomaColor_ChapterHeading}%
\definecolor{DenKrKomaColor_SubSectionHeading}{RGB}{80,0,180}%
\definecolor{DenKrKomaColor_SubSubSectionHeading}{RGB}{80,100,180}%
\definecolor{DenKrKomaColor_ParagraphHeading}{RGB}{130,150,210}%
\definecolor{DenKrKomaColor_SubParagraphHeading}{RGB}{130,150,210}%
%
\newcommand{\DenKrKomaColorOverwriteFile}{"\DenKrSupplyRootDir/OptionalDocConfig/OverwriteProperties/typografie_colors".tex}%
\IfFileExists{\DenKrKomaColorOverwriteFile}{%
\input{\DenKrKomaColorOverwriteFile}%
}{}%
%
%________________________________________________________________________
% Nummerierungen, Counter
%========================================================================
% \setcounter{secnumdepth}{5}%Wie viele Gliederungsebenen tief Nummerierungen hinzugefügt werden
% \setcounter{tocdepth}{5}%Wie viele Ebenen tief ins Inhaltsverzeichnis übernommen werden
% - - - - - - -
% \renewcommand\thesection{\arabic{section}}
% \renewcommand\thefigure{\arabic{section}.\arabic{figure}}
\renewcommand\theequation{\arabic{chapter}.\arabic{equation}}%
% - - - - - - -
%  Regarding Document Structure
\renewcommand{\thepart}{\Alph{part}}% \Roman \Alph
%
%________________________________________________________________________
%							Für Kopf und Fußzeilen der Seiten (Alles auf KOMA basierend)
%========================================================================
% \usepackage{scrpage2} %Älter
\usepackage{scrlayer-scrpage}%
\PreventPackageFromLoading{fancyhdr}%
%------------------------------------------------------------------------
%							Font-Formatierung von Kopf und Fußzeile
% \setkomafont{Element}{Befehle}
% \addtokomafont{Element}{Befehle}
\addtokomafont{pageheadfoot}{\color{DenKrKomaColor_HeadLine}}%
\addtokomafont{pagenumber}{\color{DenKrKomaColor_PageNumber}}%
%
%________________________________________________________________________
% Chapter, Section, etc. Heading Format Definition. According to KOMA-Script
%========================================================================
% Default Values are
%\newcommand*{\partformat}{\partname~\thepart\autodot}
%\newcommand*{\chapterformat}{\mbox{\chapappifchapterprefix{\nobreakspace}\thechapter\autodot\IfUsePrefixLine{}{\enskip}}}
%\newcommand*{\sectionformat}{\thesection\autodot\enskip}
%\newcommand*{\subsectionformat}{\thesubsection\autodot\enskip}
%\newcommand*{\subsubsectionformat}{\thesubsubsection\autodot\enskip}
%\newcommand*{\paragraphformat}{\theparagraph\autodot\enskip}
%\newcommand*{\subparagraphformat}{\thesubparagraph\autodot\enskip}
%\newcommand*{\othersectionlevelsformat}[3]{#3\autodot\enskip}
% - - - - - - - -
%KOMA-Script default Heading font setting
%\sffamily\bfseries
%\setkomafont{chapter}{\sffamily\bfseries\huge}
%\setkomafont{section}{\sffamily\bfseries\Large}
%\setkomafont{subsection}{\sffamily\bfseries\large}
%\setkomafont{subsubsection}{\sffamily\bfseries\normalsize}
% - - - - - - - - - - - - - - - - - - - - - - - - - - - - - - - - - - -
% - - - - - - - - - - - - - - - - - - - - - - - - - -
% For Color-Definitions, see a few Lines above
% - - - - - - - - - - - - - - - - - - - - - - - - - -
% - - - - - - - - - - - - - - - - - - - - - - - - - - - - - - - - - - -
%\setkomafont{disposition}{\normalcolor\bfseries}%
\addtokomafont{disposition}{\color{DenKrKomaColor_ChapterHeading}}%
% - - - - - - - -
\renewcommand*{\partformat}{%
    \color{DenKrKomaColor_PartHeading}\partname~\thepart\autodot\\%
    \textcolor{gray!50}{\rule[-0.2\baselineskip]{0.4\textwidth}{0.3ex}}%
}%
\setkomafont{part}{\sffamily\bfseries\huge\color{DenKrKomaColor_PartHeading}}%
% - - - - - - - -
\renewcommand*{\chapterformat}{%
    \thechapter\enskip%
    \textcolor{gray!50}{\rule[-\dp\strutbox]{2pt}{\baselineskip}}\enskip%
}%
\setkomafont{chapter}{\sffamily\bfseries\huge\color{DenKrKomaColor_ChapterHeading}}%
% - - - - - - - -
\setkomafont{section}{\sffamily\bfseries\Large\color{DenKrKomaColor_SectionHeading}}%
% - - - - - - - -
\setkomafont{subsection}{\sffamily\bfseries\large\color{DenKrKomaColor_SubSectionHeading}}%
% - - - - - - - -
% \renewcommand*{\subsubsectionformat}{%
%     \thesubsubsection\enskip%
%     \textcolor{gray!50}{\rule[-\dp\strutbox]{1em}{2pt}}\enskip%
% }%
\setkomafont{subsubsection}{\normalsize\color{DenKrKomaColor_SubSubSectionHeading}%
    \textcolor{gray!50}{\rule[0.6ex]{1em}{0.3ex}}\enskip%
    \normalsize\color{DenKrKomaColor_SubSubSectionHeading}%
}% Sehr unsauber, Textausgabe im Font zu ergänzen
% - - - - - - - -
\setkomafont{paragraph}{\normalsize\color{DenKrKomaColor_ParagraphHeading}}%
% - - - - - - - -
\setkomafont{subparagraph}{\normalsize\itshape\color{DenKrKomaColor_SubParagraphHeading}}%
%/\/\/\/\/\/\/\/\/\/\/\/\/\/\/\/\/\/\/\/\/\/\/\/\/\/\/\/\/\/\/\/\/\/\/\/\
%							Dokument Formatierungen fertig
%------------------------------------------------------------------------
%________________________________________________________________________
%
%
%
\usepackage{scrhack}%