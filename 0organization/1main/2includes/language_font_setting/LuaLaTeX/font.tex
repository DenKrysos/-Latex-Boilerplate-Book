
%________________________________________________________________________
%------------------------------------------------------------------------
%							Font/Schriftarten Setup
%/\/\/\/\/\/\/\/\/\/\/\/\/\/\/\/\/\/\/\/\/\/\/\/\/\/\/\/\/\/\/\/\/\/\/\/\
% 								%##########################
% 								% Old `fontenc'-Stuff, pre Lua(La)Tex
% 								%##########################
% 								%|=======================||||
% 								% \usepackage[T1]{fontenc}%|| Alt, pre-Lua(La)Tex
% 								%|=======================||||
% 								% % %\usepackage[scaled]{uarial}
% 								% % % \usepackage{bookman}
% 								%|===================||||
% 								% \usepackage{lmodern}%|| Lädt Latin Modern Font und setzt es fürs Dokument
% 								%|===================||||
% 								% % %Ist weniger Pixelig in pdf als Latex Standard
% 								%|====================||||
% 								% \usepackage{textcomp}%|| Lädt Text Companion Font
% 								%|====================||||
% 								% % %Stellt insbesondere Zeichen zur verfügung, wie
% 								% % %baht, bul­let, copy­right, mu­si­cal­note, onequar­ter, sec­tion, and yen
% 								% % %------------------------------------------------
% 								% % % Standardschriftart festlegen:
% 								% % % Mögliche Werte
% 								% % % \rmdefault - Roman (Serifen) Font
% 								% % % \sfdefault - Sans Serif Font
% 								% % % \ttdefault - TypeWriter Font
% 								%|=========================================||||
% 								% \renewcommand*{\familydefault}{\rmdefault}%||
% 								%|=========================================||||
% 								% % %Die drei Schriftfamilien einstellen:
% 								%|==============================||||
% 								% \renewcommand*{\rmdefault}{lmr}%||
% 								%|==============================||||
% 								% % % Standardmäßig verfügbare Schriftarten:
% 								% % % cmr 	Computer Modern Roman (default)
% 								% % % lmr 	Latin Modern Roman
% 								% % % pbk 	Bookman
% 								% % % bch 	Charter
% 								% % % pnc 	New Century Schoolbook
% 								% % % ppl 	Palatino
% 								% % % ptm 	Times
% 								%|===============================||||
% 								% \renewcommand*{\sfdefault}{lmss}%||
% 								%|===============================||||
% 								% % % Standardmäßig verfügbare Schriftarten:
% 								% % % cmss 	Computer Modern Sans Serif (default)
% 								% % % lmss 	Latin Modern Sans Serif
% 								% % % pag 	Avant Garde
% 								% % % phv 	Helvetica
% 								%|===============================||||
% 								% \renewcommand*{\ttdefault}{lmtt}%||
% 								%|===============================||||
%####################################
% Now: 'fontspec'
%####################################
\usepackage{fontspec}%
% - - - - Fonts - - - 
%  --  --  The lower, the better i like them
% \setmainfont{Times New Roman}%
% \setmainfont{Gungsuh}%
% \setmainfont{Malgun}%
% \setmainfont{Batang}%
% \setmainfont{Meiryo}%
% \setmainfont{MS Mincho}%
% \setmainfont{MS Gothic}%
% \setmainfont{NSimSun}%
% \setmainfont{SimSun}%
% \setmainfont{Kozuka Gothic Pro}% From Adobe, Downloadable free at: http://fontpark.net/de/schriftart/kozuka-gothic-pro-b/
% \setmainfont{Meiryo UI}%
% \setmainfont{Yu Gothic}%
%		\setmainfont{Yu Gothic Light}%
%		\setmainfont{Yu Gothic Medium}%
% \setmainfont{Yu Gothic UI}%
%		\setmainfont{Yu Gothic UI Light}%
%		\setmainfont{Yu Gothic UI Semibold}%
%		\setmainfont{Yu Gothic UI Semilight}%
% - - - - - - - - - - - - - - - - - - - - - - - - - - - - - - -
% \setmainfont[ItalicFont={Malgun},Ligatures=TeX]{Yu Gothic UI}% Includes Japanese Support, but not separated.
% 		Like:	% 	\setmainjfont{Yu Gothic UI} % \mcfamily
% 		 +		% 	\setsansjfont{Malgun} % \gtfamily
% \setmainfont{Times New Roman}%
% - - - - - - - - - - - - - - - - - - - - - - - - - - - - - - -
% For Japanese Characters
\usepackage{luatexja}
\usepackage{luatexja-fontspec}
% \setmainjfont{Yu Gothic UI}% \mcfamily
% \setmainjfont{Meiryo UI}[Path=./0organization/1main/2includes/fonts/]% \mcfamily
% \setsansjfont{Malgun}[Path=./0organization/1main/2includes/fonts/]% \gtfamily
\setmainjfont[Path=\DenKrLayoutMainRootDir/2includes/fonts/]{HanaMinA}% \mcfamily  % Font available for free download and installation on system "Hanazono Mincho". (Copy here in "./0organization/1main/2includes/fonts/")
\setsansjfont{Harano Aji Mincho}% \gtfamily  % Font available via Latex-Package "haranoaji" (Harano Aji Fonts)
%/\/\/\/\/\/\/\/\/\/\/\/\/\/\/\/\/\/\/\/\/\/\/\/\/\/\/\/\/\/\/\/\/\/\/\/\
%							Font/Schriftarten done
%------------------------------------------------------------------------
%________________________________________________________________________
%