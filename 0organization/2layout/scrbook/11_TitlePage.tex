%
\input{"\DenKrSupplyRootDir/TitlePage_Data".tex}%
%
%================================================================
%     Setting some Colors
%----------------------------------------------------------------
\colorlet{TitlePage_TypoAccentColorOne}{white}% A Default-Val, might be overwritten inside a Titlepage % {white} {white!60!gray} {TitlePage_colorOne}
\colorlet{TitlePage_Color_RankA}{DenKrKomaColor_PartHeading}%
\colorlet{TitlePage_Color_RankB}{DenKrKomaColor_ChapterHeading}%
\colorlet{TitlePage_Color_RankC}{DenKrKomaColor_SectionHeading}%
\colorlet{TitlePage_Color_RankD}{DenKrKomaColor_SubSectionHeading}%
%----------------------------------------------------------------
%
%
%================================================================
%----------------------------------------------------------------
%      Title Page \& Mock-Title
%================================================================
%
%\maketitle
\input{"\DenKrLayoutRootDir/supplement/title_TikZ".tex}% title_TikZ, titleDissertationEn, title_plain, titleTUKL
%
% Title Page Rückseite
\input{"\DenKrLayoutRootDir/supplement/titleMocktitlePlain".tex}% titleMocktitleEn, titleMocktitlePlain
%
%
%
% Again in German
%================================================================
%----------------------------------------------------------------
%      Title Page \& Schmutztitel
%================================================================
%
%\maketitle
\input{"\DenKrLayoutRootDir/supplement/titleDissertationGer".tex}%
%
% Title Page Rückseite
\input{"\DenKrLayoutRootDir/supplement/titleSchmutztitelGer".tex}%