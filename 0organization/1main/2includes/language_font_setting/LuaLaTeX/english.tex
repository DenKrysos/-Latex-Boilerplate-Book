%
\newcommand{\DenKrLanguageStringCmd}{english}
%________________________________________________________________________
%------------------------------------------------------------------------
%					Basic Language depending Packages
%/\/\/\/\/\/\/\/\/\/\/\/\/\/\/\/\/\/\/\/\/\/\/\/\/\/\/\/\/\/\/\/\/\/\/\/\
% \usepackage[english,german,ngerman]{babel}%
\usepackage[babelshorthands]{polyglossia}%
\setmainlanguage[variant=british]{english}% Synonym: \setdefaultlanguage
\setotherlanguage[spelling=new]{german}%
%\selectlanguage{japanese}%
% - - - - - - - - - - - - - - - - - - - - - - - - - - - - - - - - - - -
% == Note ==  With csquotes.sty, use
%     \textquote{}, to automatically print correct quotation marks, depending on the set language by polyglossia (or babel)
%     \enquote{}, for 'nested' quotations
%\usepackage[autostyle=true,german=quotes]{csquotes}% Deutsche Anführungszeichen, german=quotes, guillemets, swiss
\usepackage[autostyle=true,style=english]{csquotes}%
% - - - - - - - - - - - - - -
% == Info, »ALT-Codes« == Some specific quotation signs for hardcoding, using UTF-8 encoding. How-To-Input, if you like.
%    ALT(left) + [Sequence on NumPad]  (On Windows) (Using ASCII Decimal Index)
%      "  -    34  (Same as SHIFT + 2, Quotation Mark)
%      »  -   175  (Guillemets, Chevrons. German: Open. Swiss,French: Close)
%      «  -   174  (Guillemets, Chevrons. German: Close. Swiss,French: Open)
%      ›  -  0155
%      ‹  -  0139
%      „  -  0132  (low curly doublequote)
%      “  -  0147  (curly double open quote)
%      ”  -  0148  (curly double close quote)
%      ‘  -  0145  (curly single open quote)
%      ’  -  0146  (curly single close quote)
%      ‚  -  0130  (curly single quote)
% - - - - - - - - - - - - - -
%/\/\/\/\/\/\/\/\/\/\/\/\/\/\/\/\/\/\/\/\/\/\/\/\/\/\/\/\/\/\/\/\/\/\/\/\
%							END Basic Language
%------------------------------------------------------------------------
%________________________________________________________________________