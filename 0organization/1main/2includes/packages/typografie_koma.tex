%________________________________________________________________________
%------------------------------------------------------------------------
%							Dokument Formatierungen
%/\/\/\/\/\/\/\/\/\/\/\/\/\/\/\/\/\/\/\/\/\/\/\/\/\/\/\/\/\/\/\/\/\/\/\/\
%---------------------------------------------------------
%					Non-KOMA
%---------------------------------------------------------
% Set Page-Numbering to Capital-Roman in the Front-Matter, to Arabic in the Main-Matter and prevent both from resetting the Page-Number-Counter
%       % Remark that \pagenumbering resets the Page Counter. Hence, not done like:
%        % \renewcommand{\frontmatter}{\cleardoublepage\@mainmatterfalse\pagenumbering{Roman}}
%        % \renewcommand{\mainmatter}{\cleardoublepage\@mainmattertrue\pagenumbering{arabic}}
\makeatletter%
\renewcommand{\frontmatter}{\cleardoublepage\@mainmatterfalse\renewcommand{\thepage}{\Roman{page}}}%
\renewcommand{\mainmatter}{\cleardoublepage\@mainmattertrue\renewcommand{\thepage}{\arabic{page}}}%
\makeatother%
%
% - - - - - - - -
% Block-Stretching across page
% - - - - - - - -
%     LaTeX uses \flushbottom for two-sided documents (book by default). Odd pages and even pages are forced to be aligned. In one-sided documents (article, report by default) LaTeX uses \raggedbottom, extra spaces will be gone. cf. classes document.
%     One can use \raggedbottom if met too many bad page breaks. However, it is preferred to prevent big boxes in your document. Use floats instead of put big tabulars and figures directly. For lists and section titles, it is often not too serious, be sure you put enough text for each sections.
\raggedbottom%
%
% - - - - - - - -
% Reset the Chapter-Counter with every Part
%  (When doing this, make sure to turn the Chapter-ID (\thechapter) into something Part-specific. Similar to how it is done with Sections and below.)
% With issuing the cmd, the document is changed from this spot on. Just like \mainmatter and alike
% - - - - - - - -
% USAGE:
%  When intending to use this, also make sufficient Space in the printing of the Table-Of-Contents. For that just utilize the provided Macro below.
% -> Remove the comment-percent in Front of the pasted Macro right at the bottom of the definition
% - - - - - - - -
\makeatletter%
\newcounter{matterPartSeparationStatus}%
\setcounter{matterPartSeparationStatus}{0}%
\newcounter{PartSeparationChapterBkp}%
\newcommand{\matterPartSeparatedActivate}{%
    \ifnumcomp{\value{matterPartSeparationStatus}}{>}{0}{}{%
        \setcounter{matterPartSeparationStatus}{1}%
        \setcounter{PartSeparationChapterBkp}{\value{chapter}}%
        \@addtoreset{chapter}{part}%
        \renewcommand{\thechapter}{\Alph{part}\arabic{chapter}}%
    }%
}%
\newcommand{\matterPartSeparatedRestore}{%
    \ifnumcomp{\value{matterPartSeparationStatus}}{>}{0}{%
        \@removefromreset{chapter}{part}%
        \setcounter{chapter}{\value{PartSeparationChapterBkp}}%
        \renewcommand{\thechapter}{\arabic{chapter}}%
        \setcounter{matterPartSeparationStatus}{0}%
    }{}%
}%
\makeatother%
\newcommand{\addSpaceToTocNumForPartSeparationMatter}{%
    \AddToHook{begindocument/before}{%
        \newlength{\partnumTOClength}%
        \setlength{\partnumTOClength}{1.5em}%
        \addtolength{\cftpartnumwidth}{\partnumTOClength}%
        \addtolength{\cftchapnumwidth}{\partnumTOClength}%
        \addtolength{\cftsecnumwidth}{\partnumTOClength}%
        \addtolength{\cftsubsecnumwidth}{\partnumTOClength}%
        \addtolength{\cftsubsubsecnumwidth}{\partnumTOClength}%
        \addtolength{\cftparanumwidth}{\partnumTOClength}%
        \addtolength{\cftsubparanumwidth}{\partnumTOClength}%
        \addtolength{\cftfignumwidth}{\partnumTOClength}%
        %\addtolength{\cftsubfignumwidth}{\partnumTOClength}%
        \addtolength{\cfttabnumwidth}{\partnumTOClength}%
        %\addtolength{\cftsubtabnumwidth}{\partnumTOClength}%
        \let\partnumTOClength\undefined%
    }%
}%
% \addSpaceToTocNumForPartSeparationMatter%
%
%
%
%---------------------------------------------------------
%					KOMA
%---------------------------------------------------------
\KOMAoptions{%
 	%BCOR=8mm,% Bindeverlust von 8mm am Innenrand einbeziehen
	DIV=last,%
	open=right% any, right, left  % 'openany' at \documentclass
}%
%
%________________________________________________________________________
% Colors
% - (Are defined in the File "\DenKrLayoutMainRootDir/2includes/3settings/typografie_color.tex", which is included right before this here)
%========================================================================
%
%
%________________________________________________________________________
% Nummerierungen, Counter
%========================================================================
% \setcounter{secnumdepth}{5}%Wie viele Gliederungsebenen tief Nummerierungen hinzugefügt werden
% \setcounter{tocdepth}{5}%Wie viele Ebenen tief ins Inhaltsverzeichnis übernommen werden
% - - - - - - -
% \renewcommand\thesection{\arabic{section}}
% \renewcommand\thefigure{\arabic{section}.\arabic{figure}}
\renewcommand\theequation{\arabic{chapter}.\arabic{equation}}%
% - - - - - - -
%  Regarding Document Structure
\renewcommand{\thepart}{\Alph{part}}% \Roman \Alph
%
%________________________________________________________________________
%			Pagestyle & Layers (For Header, Footer (Kopf- & Fußzeilen) of the Pages (Everything based on KOMA))
%========================================================================
% \usepackage{scrpage2} %Älter
\usepackage{scrlayer-scrpage}%
\PreventPackageFromLoading{fancyhdr}%
%------------------------------------------------------------------------
%							Font-Formatierung von Kopf und Fußzeile
% \setkomafont{Element}{Befehle}
% \addtokomafont{Element}{Befehle}
\addtokomafont{pageheadfoot}{\color{DenKrKomaColor_HeadLine}}%
\addtokomafont{pagenumber}{\color{DenKrKomaColor_PageNumber}}%
%------------------------------------------------------------------------
% Now it is getting interesting. We are going into detail, define pagestyles and layers and make them dynamically adjustable throughout the Document
\DeclareNewLayer[%
    foot,%
    %align=t,%
    foreground,%
    mode=text,%
    addhoffset=0pt,%
    addvoffset=0pt,%
    contents={%
        % \putLL{LL}%
        % \putLR{LR}%
        % \putUR{UR}%
        % \putUL{UL}%
        % \putC{C}%
        % \putLL{\line(1,0){\LenToUnit{\layerwidth}}}%
        % \putLR{\line(0,1){\LenToUnit{\layerheight}}}%
        % \putUR{\line(-1,0){\LenToUnit{\layerwidth}}}%
        % \putUL{\line(0,-1){\LenToUnit{\layerheight}}}%
        % \putC{\circle{\LenToUnit{\paperwidth}}}%
    }%
]{DenKrKoma_Layer_Dynamic}%
\DeclareNewLayer[%
    foot,%
    twoside,%
    foreground,%
    mode=picture,%
    addhoffset=0pt,%
    addvoffset=0pt,%
    contents={}%
]{DenKrKoma_Layer_Dynamic_Twoside}%
\DeclareNewLayer[%
    foot,%
    oneside,%
    foreground,%
    mode=text,%
    addhoffset=0pt,%
    addvoffset=0pt,%
    contents={}%
]{DenKrKoma_Layer_Dynamic_Oneside}%
\newcommand{\DKDynamicFootAdd}[1]{%
    \AddLayersToPageStyle{\currentpagestyle}{DenKrKoma_Layer_Dynamic_Twoside}%
    \AddLayersToPageStyle{\currentpagestyle}{DenKrKoma_Layer_Dynamic_Oneside}%
    \settowidth{\tmpwidth}{#1}%
    %\RedeclareLayer or \ModifyLayer
    \ModifyLayer[%
        contents={%
            \ifodd\value{page}%
                \put(\LenToUnit{0.7\layerwidth-\tmpwidth},\LenToUnit{0.5\layerheight}){#1}%
            \else%
                \put(\LenToUnit{0.3\layerwidth},\LenToUnit{0.5\layerheight}){#1}%
            \fi%
        }%
    ]{DenKrKoma_Layer_Dynamic_Twoside}%
    \ModifyLayer[%
        contents={%
            #1%
        }%
    ]{DenKrKoma_Layer_Dynamic_Oneside}%
    % \ModifyLayerPageStyleOptions{\currentpagestyle}{%
    %     ontwoside={%
    %         \ModifyLayer[%
    %             contents={%
    %                 TWOSIDE%
    %             }%
    %         ]{DenKrKoma_Layer_Dynamic}%
    %     }%
    % }%
}%
\newcommand{\DKDynamicFootRem}{%
    \RemoveLayersFromPageStyle{\currentpagestyle}{DenKrKoma_Layer_Dynamic_Twoside}%
    \RemoveLayersFromPageStyle{\currentpagestyle}{DenKrKoma_Layer_Dynamic_Oneside}%
}%
% USAGE: In some cases, you might want to use '\DKDynamicFootRem' or '\DKDynamicFootAdd{}' together with 'afterpage':
%  \afterpage{\DKDynamicFootRem}
%________________________________________________________________________
%           Pagestyle DONE
%------------------------------------------------------------------------
%========================================================================
%
%
%
%________________________________________________________________________
%           Chapter, Section, etc. Heading Format Definition. According to KOMA-Script
%========================================================================
% Default Values are
%\newcommand*{\partformat}{\partname~\thepart\autodot}
%\newcommand*{\chapterformat}{\mbox{\chapappifchapterprefix{\nobreakspace}\thechapter\autodot\IfUsePrefixLine{}{\enskip}}}
%\newcommand*{\sectionformat}{\thesection\autodot\enskip}
%\newcommand*{\subsectionformat}{\thesubsection\autodot\enskip}
%\newcommand*{\subsubsectionformat}{\thesubsubsection\autodot\enskip}
%\newcommand*{\paragraphformat}{\theparagraph\autodot\enskip}
%\newcommand*{\subparagraphformat}{\thesubparagraph\autodot\enskip}
%\newcommand*{\othersectionlevelsformat}[3]{#3\autodot\enskip}
% - - - - - - - -
%KOMA-Script default Heading font setting
%\sffamily\bfseries
%\setkomafont{chapter}{\sffamily\bfseries\huge}
%\setkomafont{section}{\sffamily\bfseries\Large}
%\setkomafont{subsection}{\sffamily\bfseries\large}
%\setkomafont{subsubsection}{\sffamily\bfseries\normalsize}
% - - - - - - - - - - - - - - - - - - - - - - - - - - - - - - - - - - -
% - - - - - - - - - - - - - - - - - - - - - - - - - -
% For Color-Definitions, see a few Lines above
% - - - - - - - - - - - - - - - - - - - - - - - - - -
%  Defining some Symbols
\newcommand*{\DenKrKomaHeadingAsteriskChar}{{\small$\mathrm{\ast}$}}%
\newlength{\DenKrKomaHeadingAsteriskRaise}%
\setlength{\DenKrKomaHeadingAsteriskRaise}{\heightof{\vphantom{H}}-\heightof{\DenKrKomaHeadingAsteriskChar}}%
\setlength{\DenKrKomaHeadingAsteriskRaise}{0.5\DenKrKomaHeadingAsteriskRaise}%
\newcommand*{\DenKrKomaHeadingAsterisk}{\raisebox{\DenKrKomaHeadingAsteriskRaise}{\DenKrKomaHeadingAsteriskChar}}%
% - - - - - - - - - - - - - - - - - - - - - - - - - -
% - - - - - - - - - - - - - - - - - - - - - - - - - - - - - - - - - - -
%\setkomafont{disposition}{\normalcolor\bfseries}%
\addtokomafont{disposition}{\color{DenKrKomaColor_ChapterHeading}}%
% - - - - - - - -
\renewcommand*{\partformat}{%
    \color{DenKrKomaColor_PartHeading}\partname~\thepart\autodot\\%
    \textcolor{gray!50}{\rule[-0.2\baselineskip]{0.4\textwidth}{0.3ex}}%
}%
\setkomafont{part}{\sffamily\bfseries\huge\color{DenKrKomaColor_PartHeading}}%
% - - - - - - - -
\renewcommand*{\chapterformat}{%
    \thechapter\enskip%
    \textcolor{gray!50}{\rule[-\dp\strutbox]{2pt}{\baselineskip}}\enskip%
}%
\setkomafont{chapter}{\sffamily\bfseries\huge\color{DenKrKomaColor_ChapterHeading}}%
% - - - - - - - -
\setkomafont{section}{\sffamily\bfseries\Large\color{DenKrKomaColor_SectionHeading}}%
% - - - - - - - -
\setkomafont{subsection}{\sffamily\bfseries\large\color{DenKrKomaColor_SubSectionHeading}}%
% - - - - - - - -
% \renewcommand*{\subsubsectionformat}{%
%     \thesubsubsection\enskip%
%     \textcolor{gray!50}{\rule[-\dp\strutbox]{1em}{2pt}}\enskip%
% }%
\setkomafont{subsubsection}{\normalsize\color{DenKrKomaColor_SubSubSectionHeading}%
    \textcolor{gray!50}{\rule[0.6ex]{1em}{0.3ex}}\enskip%
    \normalsize\color{DenKrKomaColor_SubSubSectionHeading}%
}% Sehr unsauber, Textausgabe im Font zu ergänzen
% - - - - - - - -
\RedeclareSectionCommands[
    runin=bysign,
    afterskip=-0.5em,
    % afterindent=bysign,
    % beforeskip=0pt,
]{paragraph}
\AddtoDoHook{heading/endgroup/paragraph}{\DenKrKomaHeadingParagraphAttach}%
\newcommand*{\DenKrKomaHeadingParagraphAttach}[1]{\hspace{0.5em}\textcolor{DenKrKomaColor_SubSectionHeading}{\DenKrKomaHeadingAsterisk}}%
\setkomafont{paragraph}{\normalsize\color{DenKrKomaColor_ParagraphHeading}}%
% - - - - - - - -
\RedeclareSectionCommands[
    runin=bysign,
    afterskip=-0.4em
]{subparagraph}
\AddtoDoHook{heading/endgroup/subparagraph}{\DenKrKomaHeadingSubParagraphAttach}%
\newcommand*{\DenKrKomaHeadingSubParagraphAttach}[1]{{\normalfont\hspace{0.5em}\textcolor{DenKrKomaColor_SubSectionHeading}{\textbf{-}}}}%
\setkomafont{subparagraph}{\normalsize\itshape\color{DenKrKomaColor_SubParagraphHeading}}%
%________________________________________________________________________
%           Sectioning DONE
%------------------------------------------------------------------------
%========================================================================
% NOTE: Alternative Solution to change the formatting of a Section-Heading-Name:
%        \renewcommand*\sectioncatchphraseformat[4]{%
%           \hskip #2#3#4\Ifstr{#1}{paragraph}{.}{}%
%        }
%
%
%________________________________________________________________________
%			Some Commands/Macros
%========================================================================
%                \makeatletter
%                \setlength{\tmpwidth}{\scr@paragraph@beforeskip}
%                \makeatother
%                Current beforeskip for paragraph: \the\tmpwidth
%                % - - - - - - - - - - - - -
%                \setCollectVar{someCollection}{Var1}{\tmpwidth}
%                \setlength{\tmpwidth}{\getCollectVar{someCollection}{Var1}}
%                Set-Read: \the\tmpwidth
%                % - - - - - - - - - - - - -
%                % \setCollectVar{someCollection}{Var1}{\tmpwidth}
%                % Value is: \getCollectVar{someCollection}{Var1}
%
% Allows to start a single Paragraph/etc. without having a whitespace / vertical Skip before the Heading.
%   It (temporarily) removes the Beforeskip.
%   Is supposed to be used together with the "Restore" Macro.
%  USAGE:
%         \headingBeforeskipTempRem{paragraph}% Results in no beforeskip for all \paragraph{} afterwards
%         % Do some stuff, i.e. begin \paragraph{}s
%         \headingBeforeskipRestore{paragraph}% Restores the previous setting
%         
% ARG: Gets (as a simple string) the sectioning-Level, for which it shall remove/restore the Beforeskip
\makeatletter%
\newcommand{\headingBeforeskipTempSet}[2]{% 1: The Sectioning-Level  2: The length to set to
    \setlength{\tmplen}{\csname scr@#1@beforeskip\endcsname}%
    %\npi Current beforeskip for #1: \the\tmplen.\nl%
    \setCollectVar{prevBeforeskip}{#1}{\tmplen}%
    \RedeclareSectionCommands[%
        beforeskip=#2%
    ]{#1}%
}%
\newcommand{\headingBeforeskipTempRem}[1]{%
    \headingBeforeskipTempSet{#1}{0pt}%
}%
\newcommand{\headingBeforeskipRestore}[1]{%
    \setlength{\tmplen}{\getCollectVar{prevBeforeskip}{#1}}%
    %\npi Read previous Val from tmpStorage: \the\tmpwidth.\nl%
    \RedeclareSectionCommands[%
        beforeskip=\tmplen%
    ]{#1}%
    %\npi\setlength{\tmplen}{\csname scr@#1@beforeskip\endcsname}Again set beforeskip for #1: \the\tmplen.\nl%
    \destroyCollectVar{prevBeforeskip}{#1}%
}%
\makeatother%
%
%
% For Setting, Storing, Restoring and Checking whether openany, openright, openleft
% USAGE:  (The \makeatletter wrapper is actually optional)
%            \makeatletter\headingOpenTempRight%
%            \part{Thesis}%
%            \headingOpenTempRestore\makeatother%
%            \label{part:thesis}%
\makeatletter%
\newcommand{\currentOpenAtCheck}{%
    \if@openright%
        RIGHT%
    \else%
        NOT-Right%
    \fi%
}%
\newcommand{\recordCurrentOpen}{%
    % Note: It does not properly check whether openLEFT is set. I am not completely sure, but I think, openleft actually only sets openright and changes the behavior of some \cleardouble[...] cmds.
    \if@openright%
        \setCollectVar{prevOpen}{Heading}{2}% right
    \else%
        \setCollectVar{prevOpen}{Heading}{1}% any
    \fi%
}%
\newcommand{\headingOpenTempRight}{%
    \recordCurrentOpen%
    \@openrighttrue%
}%
\newcommand{\headingOpenTempAny}{%
    \recordCurrentOpen%
    \@openrightfalse%
}%
\newcommand{\headingOpenTempRestore}{%
    \ifnumcomp{\expandafter\getCollectVar{prevOpen}{Heading}}{=}{2}{%
        \@openrighttrue%
    }{%
        \@openrightfalse%
    }%
    \destroyCollectVar{prevOpen}{Heading}%
}%
\makeatother%
%________________________________________________________________________
%           Macros DONE
%------------------------------------------------------------------------
%========================================================================
%
%
%/\/\/\/\/\/\/\/\/\/\/\/\/\/\/\/\/\/\/\/\/\/\/\/\/\/\/\/\/\/\/\/\/\/\/\/\
%							Dokument Formatierungen fertig
%------------------------------------------------------------------------
%________________________________________________________________________
%
%
%
\usepackage{scrhack}%