%________________________________________________________________________
%------------------------------------------------------------------------
%							Dokument Formatierungen
%/\/\/\/\/\/\/\/\/\/\/\/\/\/\/\/\/\/\/\/\/\/\/\/\/\/\/\/\/\/\/\/\/\/\/\/\
%---------------------------------------------------------
%					Non-KOMA
%---------------------------------------------------------
% Set Page-Numbering to Capital-Roman in the Front-Matter, to Arabic in the Main-Matter and prevent both from resetting the Page-Number-Counter
%       % Remark that \pagenumbering resets the Page Counter. Hence, not done like:
%        % \renewcommand{\frontmatter}{\cleardoublepage\@mainmatterfalse\pagenumbering{Roman}}
%        % \renewcommand{\mainmatter}{\cleardoublepage\@mainmattertrue\pagenumbering{arabic}}
\makeatletter%
\renewcommand{\frontmatter}{\cleardoublepage\@mainmatterfalse\renewcommand{\thepage}{\Roman{page}}}%
\renewcommand{\mainmatter}{\cleardoublepage\@mainmattertrue\renewcommand{\thepage}{\arabic{page}}}%
\makeatother%
%
% - - - - - - - -
% Block-Stretching across page
% - - - - - - - -
%     LaTeX uses \flushbottom for two-sided documents (book by default). Odd pages and even pages are forced to be aligned. In one-sided documents (article, report by default) LaTeX uses \raggedbottom, extra spaces will be gone. cf. classes document.
%     One can use \raggedbottom if met too many bad page breaks. However, it is preferred to prevent big boxes in your document. Use floats instead of put big tabulars and figures directly. For lists and section titles, it is often not too serious, be sure you put enough text for each sections.
\raggedbottom%
%
%
%
%---------------------------------------------------------
%					KOMA
%---------------------------------------------------------
\KOMAoptions{%
% 	BCOR=8mm,% Bindeverlust von 8mm am Innenrand einbeziehen
	DIV=last,%
}%
%
%________________________________________________________________________
% Nummerierungen, Counter
%========================================================================
% \setcounter{secnumdepth}{5}%Wie viele Gliederungsebenen tief Nummerierungen hinzugefügt werden
% \setcounter{tocdepth}{5}%Wie viele Ebenen tief ins Inhaltsverzeichnis übernommen werden
% - - - - - - -
% \renewcommand\thesection{\arabic{section}}
% \renewcommand\thefigure{\arabic{section}.\arabic{figure}}
\renewcommand\theequation{\arabic{chapter}.\arabic{equation}}%
%
%________________________________________________________________________
%							Für Kopf und Fußzeilen der Seiten (Alles auf KOMA basierend)
%========================================================================
% \usepackage{scrpage2} %Älter
\usepackage{scrlayer-scrpage}%
\PreventPackageFromLoading{fancyhdr}%
%------------------------------------------------------------------------
%							Font-Formatierung von Kopf und Fußzeile
% \setkomafont{Element}{Befehle}
% \addtokomafont{Element}{Befehle}
\definecolor{HeadLineColor}{RGB}{40,60,150}%
\definecolor{FootLineColor}{named}{HeadLineColor}%
\addtokomafont{pageheadfoot}{\color{HeadLineColor}}%
\definecolor{PageNumberColor}{RGB}{0,60,150}%
\addtokomafont{pagenumber}{\color{HeadLineColor}}%
%
%________________________________________________________________________
% Chapter, Section, etc. Heading Format Definition. According to KOMA-Script
%========================================================================
% Default Values are
%\newcommand*{\partformat}{\partname~\thepart\autodot}
%\newcommand*{\chapterformat}{\mbox{\chapappifchapterprefix{\nobreakspace}\thechapter\autodot\IfUsePrefixLine{}{\enskip}}}
%\newcommand*{\sectionformat}{\thesection\autodot\enskip}
%\newcommand*{\subsectionformat}{\thesubsection\autodot\enskip}
%\newcommand*{\subsubsectionformat}{\thesubsubsection\autodot\enskip}
%\newcommand*{\paragraphformat}{\theparagraph\autodot\enskip}
%\newcommand*{\subparagraphformat}{\thesubparagraph\autodot\enskip}
%\newcommand*{\othersectionlevelsformat}[3]{#3\autodot\enskip}
% - - - - - - - -
%KOMA-Script default Heading font setting
%\sffamily\bfseries
%\setkomafont{chapter}{\sffamily\bfseries\huge}
%\setkomafont{section}{\sffamily\bfseries\Large}
%\setkomafont{subsection}{\sffamily\bfseries\large}
%\setkomafont{subsubsection}{\sffamily\bfseries\normalsize}
% - - - - - - - - - - - - - - - - - - - - - - - - - - - - - - - - - - -
% - - - - - - - - - - - - - - - - - - - - - - - - - -
\definecolor{PartHeadingColor}{RGB}{40,0,100}%
\definecolor{ChapterHeadingColor}{RGB}{60,0,150}%
\definecolor{SectionHeadingColor}{named}{ChapterHeadingColor}%
\definecolor{SubSectionHeadingColor}{RGB}{80,0,180}%
\definecolor{SubSubSectionHeadingColor}{RGB}{80,100,180}%
\definecolor{ParagraphHeadingColor}{RGB}{130,150,210}%
\definecolor{SubParagraphHeadingColor}{RGB}{130,150,210}%
% - - - - - - - - - - - - - - - - - - - - - - - - - -
% - - - - - - - - - - - - - - - - - - - - - - - - - - - - - - - - - - -
%\setkomafont{disposition}{\normalcolor\bfseries}%
\addtokomafont{disposition}{\color{ChapterHeadingColor}}%
% - - - - - - - -
\renewcommand*{\partformat}{%
    \color{PartHeadingColor}\partname~\thepart\autodot\\%
    \textcolor{gray!50}{\rule[-0.2\baselineskip]{0.4\textwidth}{0.3ex}}%
}%
\setkomafont{part}{\sffamily\bfseries\huge\color{PartHeadingColor}}%
% - - - - - - - -
\renewcommand*{\chapterformat}{%
    \thechapter\enskip%
    \textcolor{gray!50}{\rule[-\dp\strutbox]{2pt}{\baselineskip}}\enskip%
}%
\setkomafont{chapter}{\sffamily\bfseries\huge\color{ChapterHeadingColor}}%
% - - - - - - - -
\setkomafont{section}{\sffamily\bfseries\Large\color{SectionHeadingColor}}%
% - - - - - - - -
\setkomafont{subsection}{\sffamily\bfseries\large\color{SubSectionHeadingColor}}%
% - - - - - - - -
\renewcommand*{\subsubsectionformat}{%
    \thesubsubsection\enskip%
    \textcolor{gray!50}{\rule[-\dp\strutbox]{1em}{2pt}} af\enskip%
}%
\setkomafont{subsubsection}{\normalsize\color{SubSubSectionHeadingColor}%
    \textcolor{gray!50}{\rule[0.6ex]{1em}{0.3ex}}\enskip%Sehr unsauber, Textausgabe im Font zu ergänzen
}%
% - - - - - - - -
\setkomafont{paragraph}{\normalsize\color{ParagraphHeadingColor}}%
% - - - - - - - -
\setkomafont{subparagraph}{\normalsize\itshape\color{SubParagraphHeadingColor}}%
%/\/\/\/\/\/\/\/\/\/\/\/\/\/\/\/\/\/\/\/\/\/\/\/\/\/\/\/\/\/\/\/\/\/\/\/\
%							Dokument Formatierungen fertig
%------------------------------------------------------------------------
%________________________________________________________________________
%
%
%
\usepackage{scrhack}%