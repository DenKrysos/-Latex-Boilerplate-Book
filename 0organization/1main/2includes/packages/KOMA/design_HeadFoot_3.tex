%
% This one 'only' adds a Background and then simply includes "HeadFoot_2"
%
%########################################################################
%========================================================================
%       Background
%________________________________________________________________________
%------------------------------------------------------------------------
%
%
%
%
%========================================================================
%       Setting up some Macros for late use
% - - - - - - - - - - - - - - - - - - - - - - - - - - - - - - - - - - - -
\newcommand{\HeadFootShadeTriangleB}[4]{% 1: BL  2: BR  3: Top  4: Steps
    \tikzmath{%
        \opacstep=1/(#4+1);%
        \scalestep=1/#4;%
        \scalecur=1;
    }%
    \foreach\i in {1,...,#4}{%
        \tikzmath{%
            \smoothen=1/(sqrt(\i));%
            \opacCur=\i*\opacstep*\smoothen;%
            \scalecur=(\i-1)*\scalestep;%
        }%
        \path[fill=FooterBG,line width=0pt,opacity=\opacCur]let\p1=($(#1)!\scalecur!(#2)$),\p2=($(#3)!\scalecur!(#2)$)in%
            (\p1)--(#2)--(\p2)--cycle;%
    }%
    \let\scalestep\undefined\let\opacstep\undefined\let\scalecur\undefined\let\smoothen\undefined\let\opacCur\undefined%
}%
\newcommand{\HeadFootShadeTriangleBR}[3]{%
    \HeadFootShadeTriangleB{#1}{#2}{#3}{20}%
}%
\newcommand{\HeadFootShadeTriangleBL}[3]{%
    \HeadFootShadeTriangleB{#2}{#1}{#3}{20}%
}%
%
\newcommand{\HeadFootShadeTriangleBInverse}[4]{% 1: L  2: R  3: Top  4: Steps
    \tikzmath{%
        \opacstep=1/(#4+1);%
        \scalestep=1/#4;%
        \scalecur=1;%
    }%
    \foreach\i in {1,...,#4}{%
        \tikzmath{%
            \smoothen=1/(sqrt(\i));%
            \opacCur=\i*\opacstep*\smoothen;%
            \scalecur=(\i-1)*\scalestep;%
        }%
        \path[fill=FooterBG,line width=0pt,opacity=\opacCur]%
            let\p1=($(#3)!\scalecur!(#2)$),\p2=($(#3)!\scalecur!(#1)$)in%
            (\p1)--(#2)--(#1)--(\p2)--cycle;%
    }%
    \let\scalestep\undefined\let\opacstep\undefined\let\scalecur\undefined\let\smoothen\undefined\let\opacCur\undefined%
}%
\newcommand{\HeadFootShadeTriangleBInverseTopL}[3]{% 1: LB  2: RT  3: (L)Top
    \HeadFootShadeTriangleBInverse{#1}{#2}{#3}{20}%
}%
\newcommand{\HeadFootShadeTriangleBInverseTopR}[3]{% 1: LT  2: RB  3: (T)Top
    \HeadFootShadeTriangleBInverse{#2}{#1}{#3}{20}%
}%
%
%
%
% Not used currently, also not quite yet finished
\newcommand{\HeadFootShadeRectangleBR}[3]{% 1: TL  2: BR  3: Steps
    \tikzmath{%
        \opacstep=1/(#3+1);%
        \scalestep=1/#3;%
        \scalecur=1;%
    }%
    \foreach\i in {1,...,#3}{%
        \tikzmath{%
            \smoothen=1/(sqrt(\i));%
            \opacCur=\i*\opacstep*\smoothen;%
            \scalecur=(\i-1)*\scalestep;%
        }%
        \path[fill=orange,line width=0pt,opacity=\opacCur]%
            let\p1=(#1-|#2),\p2=(#1|-#2),%
            \p3=($(\p1)!\scalecur!(#1)$),\p4=($(\p2)!\scalecur!(#1)$)in%
            (\p3)--(\p1)|-(\p2)--(\p4)--cycle;%
    }%
    \let\scalestep\undefined\let\opacstep\undefined\let\scalecur\undefined\let\smoothen\undefined\let\opacCur\undefined%
}%
%
%
%
%
%
% NONE of the above are actually used anymore. Exclusively this one below.
\DeclareDocumentCommand{\HeadFootShadeCorner}{%
    % Define the Coordinates to use before accordingly
    %  footShadeCorner_F[x]: A Fixed-Point
    %  footShadeCorner_M[x]: A Moving-Point
    O{20}% 1: Number of Shading Steps
}{%
    \tikzmath{%
        \opacstep=1/(#1+1);%
        \scalestep=1/#1;%
        \scalecur=1;%
    }%
    \foreach\i in {0,...,#1}{%
        \tikzmath{%
            \smoothen=1/(sqrt(\i+1));%
            \opacCur=(\i+1)*\opacstep*\smoothen;%
            \scalecur=\i*\scalestep;%
        }%
        \path[fill=FooterBG,line width=0pt,opacity=\opacCur]%
            let\p1=($(footShadeCorner_BottomMov)!\scalecur!(footShadeCorner_BottomFix)$),\p2=($(footShadeCorner_Mid1Mov)!\scalecur!(footShadeCorner_Mid1Fix)$),\p3=($(footShadeCorner_Mid2Mov)!\scalecur!(footShadeCorner_Mid2Fix)$),\p4=($(footShadeCorner_TopMov)!\scalecur!(footShadeCorner_TopFix)$)in%
            (footCornerAnch)--(\p1)--(\p2)--(\p3)--(\p4)--cycle;%
    }%
    \let\scalestep\undefined\let\opacstep\undefined\let\scalecur\undefined\let\smoothen\undefined\let\opacCur\undefined%
}%
%
%
%
%
% \colorlet{HeaderFooterBase}{DenKrKomaColor_ChapterHeading!70!DenKrKomaColor_ChapterHeading_2}%
% \colorlet{HeaderBG}{DenKrKomaColor_ChapterHeading!10!white}%
% \colorlet{HeaderBG2}{HeaderFooterBase!25!white}%
% \colorlet{FooterBG}{DenKrKomaColor_ChapterHeading!15!white}%
%
\colorlet{HeaderFooterBase}{DenKrKomaColor_ChapterHeading!80!DenKrKomaColor_ChapterHeading_2}%
\colorlet{HeaderBG}{DenKrKomaColor_ChapterHeading!6!white}%
\colorlet{HeaderBG2}{HeaderFooterBase!18!white}%
\colorlet{FooterBG}{DenKrKomaColor_ChapterHeading!15!white}%
%
%
%========================================================================
%       Header
% - - - - - - - - - - - - - - - - - - - - - - - - - - - - - - - - - - - -
% A Custom Shading for the Header-Background (Remark that the Colors are defined in REVERSE Order. The Shading fills from Bottom to Top, i.e. 0bp is at the Bottom, 50bp is at the Top)
\pgfdeclareverticalshading{headerBGShading}{100bp}{%
    color(0bp)=(white);%
    color(8bp)=(HeaderBG);%
    color(22bp)=(HeaderBG2);%
    color(32bp)=(HeaderBG!70!HeaderBG2);%
    color(50bp)=(HeaderBG)%
}%
% - - - - - - - - - - - -
% The actual Layer for the Header
\DeclareNewLayer[%
  background,%
  mode=picture,%
  oddorevenpage,%
  head,%
  contents={%
    \put(0,-\LenToUnit{\layerheight}){\tikzmark{headLBAnchor}}%
    \begin{tikzpicture}[overlay,remember picture]%
    \coordinate(headBgTL)at(current page.north west);%
    \coordinate(headBgTR)at(current page.north east);%
    \path let\p1=(pic cs:headLBAnchor),\p2=($(\p1)+(0,10pt)$)in coordinate(headBgBL)at(headBgTL|-\p2);% Increase by a share of \headsep ?
    \coordinate(headBgBR)at(headBgBL-|headBgTR);%
    \path[draw=none,line width=0pt,shading=headerBGShading](headBgTL)rectangle(headBgBR);%
    \end{tikzpicture}%
  }%
]{headBG.both}%
\AddLayersToPageStyle{scrheadings}{headBG.both,headBG.both}%
%
%
%========================================================================
%       Footer
% - - - - - - - - - - - - - - - - - - - - - - - - - - - - - - - - - - - -
\DeclareNewLayer[%
  background,%
  mode=picture,%
  oddpage,%
  foot,%
  contents={%
    \put(\LenToUnit{\layerwidth},\LenToUnit{\layerheight}){\tikzmark{footRBAnchor}}%
    \begin{tikzpicture}[overlay,remember picture]%
    \coordinate(footBGAnchR)at($(pic cs:footRBAnchor)+(0,5pt)$);%
    \coordinate(footBgR_BR)at(current page.south east);%
    \coordinate(footBgR_BL)at($(footBGAnchR|-footBgR_BR)+(-0.4*\textwidth,0)$);%
    \coordinate(footBgR_RT)at($(footBGAnchR-|footBgR_BR)+(0,{0.2*(\paperwidth-\textwidth)})$);%
    \coordinate(footCornerR_TL)at(footBGAnchR);%
    \coordinate(footCornerR_TR)at(footBGAnchR-|footBgR_BR);%
    \coordinate(footCornerR_BL)at(footBGAnchR|-footBgR_BR);%
    %
    \coordinate(footCornerAnch)at(footBgR_BR);%
    \coordinate(footShadeCorner_BottomFix)at(footCornerR_BL);%
    \coordinate(footShadeCorner_BottomMov)at(footBgR_BL);%
    \coordinate(footShadeCorner_Mid1Fix)at(footShadeCorner_BottomFix);%
    \coordinate(footShadeCorner_Mid1Mov)at(footCornerR_TL);%
    \coordinate(footShadeCorner_Mid2Fix)at(footCornerR_TR);%
    \coordinate(footShadeCorner_Mid2Mov)at(footShadeCorner_Mid1Mov);%
    \coordinate(footShadeCorner_TopFix)at(footShadeCorner_Mid2Fix);%
    \coordinate(footShadeCorner_TopMov)at(footBgR_RT);%
    \HeadFootShadeCorner%
    %\HeadFootShadeTriangleBR{footBgR_BL}{footCornerR_BL}{footCornerR_TL}%
    %\HeadFootShadeTriangleBR{footCornerR_TL}{footCornerR_TR}{footBgR_RT}%
    %\HeadFootShadeTriangleBInverseTopL{footCornerR_BL}{footCornerR_TR}{footCornerR_TL}%
    %\path[draw=none,line width=0pt,fill=FooterBG](footCornerR_BL)--(footCornerR_TR)|-cycle;%
    \end{tikzpicture}%
  }%
]{footBG.odd}%
\DeclareNewLayer[%
  background,%
  mode=picture,%
  evenpage,%
  foot,%
  contents={%
    \put(0,\LenToUnit{\layerheight}){\tikzmark{footLBAnchor}}%
    \begin{tikzpicture}[overlay,remember picture]%
    \coordinate(footBGAnchL)at($(pic cs:footLBAnchor)+(0,5pt)$);%
    \coordinate(footBgL_BL)at(current page.south west);%
    \coordinate(footBgL_BR)at($(footBGAnchL|-footBgL_BL)+(0.4*\textwidth,0)$);%
    \coordinate(footBgL_LT)at($(footBGAnchL-|footBgL_BL)+(0,{0.2*(\paperwidth-\textwidth)})$);%
    \coordinate(footCornerL_TR)at(footBGAnchL);%
    \coordinate(footCornerL_TL)at(footBGAnchL-|footBgL_BL);%
    \coordinate(footCornerL_BR)at(footBGAnchL|-footBgL_BL);%
    %
    \coordinate(footCornerAnch)at(footBgL_BL);%
    \coordinate(footShadeCorner_BottomFix)at(footCornerL_BR);%
    \coordinate(footShadeCorner_BottomMov)at(footBgL_BR);%
    \coordinate(footShadeCorner_Mid1Fix)at(footShadeCorner_BottomFix);%
    \coordinate(footShadeCorner_Mid1Mov)at(footCornerL_TR);%
    \coordinate(footShadeCorner_Mid2Fix)at(footCornerL_TL);%
    \coordinate(footShadeCorner_Mid2Mov)at(footShadeCorner_Mid1Mov);%
    \coordinate(footShadeCorner_TopFix)at(footShadeCorner_Mid2Fix);%
    \coordinate(footShadeCorner_TopMov)at(footBgL_LT);%
    \HeadFootShadeCorner%
    \end{tikzpicture}%
  }%
]{footBG.even}%
\AddLayersToPageStyle{scrheadings}{footBG.odd,footBG.even}%
\AddLayersToPageStyle{plain.scrheadings}{footBG.odd,footBG.even}%
%
%
%
%
%########################################################################
%========================================================================
%       Page-Number & Headmark taken from one Index lower
%________________________________________________________________________
%------------------------------------------------------------------------
\input{"\DenKrLayoutMainRootDir/2includes/packages/KOMA/design_HeadFoot_2".tex}%