%________________________________________________________________________
%------------------------------------------------------------------------
%           Typesetting / Schriftsatz / Satz (Druck)
%/\/\/\/\/\/\/\/\/\/\/\/\/\/\/\/\/\/\/\/\/\/\/\/\/\/\/\/\/\/\/\/\/\/\/\/\
%
% - - - - - - - -
% Text Alignment
% - - - - - - - -
\usepackage{ragged2e}%
%
\usepackage{wrapfig}% Allows to have a figure with text wrapped around
\usepackage{varwidth}% Cmds for textsize relative to surrounding text (as opposed to always absolut) -> (\smaller vs \small)
%
%
%
%
%
%
%________________________________________________________________________
%------------------------------------------------------------------------
%           More matching "Typography"
%               (but common for Article & Book level classes, hence in this shared file)
% - - - - - - - - - - - - - -
% See PassOptions for xcolor on Top
\usepackage{xcolor}%
% - - - - - - - - - - - - - -
\usepackage{textcomp}% LaTeX support for the Text Companion fonts
\usepackage{ulem}\normalem%
\usepackage{relsize}% Set the font size relative to the current font size
\usepackage{xspace}% Define commands that appear not to eat spaces
\usepackage{soul}% Hyphenation for letterspacing, underlining, and more
\usepackage{setspace}%
% - - - - - - - - - - - - - -
% Creating an Outline/Contour around Text
\newcommand{\DenKrOutlineWidth}{0.03em}%
\providecommand{\DenKrCompiler}{pdfLaTeX}%
\expandafter\ifstrequal\expandafter{\DenKrCompiler}{pdfLaTeX}{%
    \usepackage[pdftex,outline]{contour}%
    \contourlength{\DenKrOutlineWidth}% Thickness of the Outline. Default: 0.03em. Not expanded on Definition, but on Use.
    \contournumber{25}% How many copies are created. Default: 16
    %USAGE Example:
    % {%
    %     \LARGE%
    %     \contour{violet}{\textcolor{orange}{Text with Outline/Contour}}\nl%
    %     {%
    %         \contourlength{0.2em}%
    %         \contour{violet}{\textcolor{orange}{Text with Outline/Contour}}\nl%
    %         \contour{violet}{\textcolor{orange}{Text with Outline/Contour}}\nl%
    %     }%
    %     \contour{violet}{\textcolor{orange}{Text with Outline/Contour}}\nl%
    % }%
}{\expandafter\ifstrequal\expandafter{\DenKrCompiler}{LuaLaTeX}{%
    \usepackage{pdfrender}%
    %USAGE Example:
    % - The logic is kind-of inverted here.
    %   - What is set as the actual 'textcolor' is actually reflected as the Color of the Contour
    %   - The \textpdfrender is "filling" this inside.
    % \definecolor{CharFillColor}{rgb}{1,.8,.8}
    % \colorlet{CharFillColor}{orange}
    % \begin{document}
    %     \sffamily
    %     \textcolor{violet}{
    %     \textpdfrender{
    %     TextRenderingMode=FillStroke,
    %     LineWidth=\DenKrOutlineWidth,
    %     FillColor=CharFillColor,
    %     MiterLimit=1,
    %     }{Text with Outline/Contour}}
    % \end{document}
}{%
}}%
% - - - - - - - - - - - - - -
% Additional Formatting-Cmds, letters, signs, special characters, units
% - - - - - - - - - - - - - -
%\usepackage{units}% Better don't use anymore. Instead siunitx.sty since it is newer and still maintained  %contains package 'nicefrac'.
\usepackage{nicefrac}% Supplies some nice Fraction formatting. \nicefrac{<Numerator>}{<Denominator>}
\usepackage{xfrac}% Another Fraction Formator. \sfrac[⟨instance⟩]{⟨num⟩}[⟨sep⟩]{⟨denom⟩}
% - - - SI Units - - - -
% When writing a fraction within a SI-Unit Statement, deactivate the number-parsing singular for this one occurence:
%   \num[parse-numbers=false]{\frac{7}{2}}
\usepackage[english,german,ngerman]{translator}%
\usepackage{siunitx}%
\sisetup{%
    input-ignore={,},%
    input-decimal-markers={.},%
    %output-decimal-marker={.},%
    %group-separator={,},%
    group-minimum-digits=5,% Determines the number of digits after which the grouping starts to be applied. Default: 5
    group-digits=all,%
    fraction-function=\nicefrac,% \dfrac
    range-phrase=-,%
    range-units=single%
}%
\DeclareSIUnit{\calorie}{cal}%
\DeclareSIUnit{\tablespoon}{\translate{tbsp}}%
\DeclareSIUnit{\teaspoon}{\translate{tsp}}%
\newtranslation[to=German]{tbsp}{EL}%
\newtranslation[to=German]{tsp}{TL}%
\addto\extrasgerman{\sisetup{locale=DE,output-decimal-marker={,},group-separator={.},detect-all}}%
\addto\extrasenglish{\sisetup{locale=US,output-decimal-marker={.},group-separator={,},detect-all}}%
% For some reason, group-separator does not work in the addto configuration, hence adding it again below
\expandafter\ifstrequal\expandafter{\DenKrLanguageStringCmd}{ngerman}{%
    \sisetup{locale=DE,group-separator={.}}%
}{\expandafter\ifstrequal\expandafter{\DenKrLanguageStringCmd}{english}{%
    \sisetup{locale=US,group-separator={,}}%
}{}}%
%
%  Example:
%        \qty[locale=DE]{2}{\tablespoon}
%        %
%        \begin{otherlanguage}{english}
%        \qty{2}{\tablespoon}
%        \end{otherlanguage}
%        %
%        \begin{otherlanguage}{german}
%        \qty{2}{\tablespoon}
%        \end{otherlanguage}
% - - - -
% Two Macros for pasting time (time-of-day or time-span). Allow an optional first argument for local set up in the usual siunitx way.
\ExplSyntaxOn
% Hour-Minute-Second-Period (For pasting a Period of Time. Like \hmsP{10;20;30} -> "10:20:30 h" | \hmsP{;10;30}  ->  "20:30 min"
\NewDocumentCommand\hmsP{o >{\SplitArgument{2}{;}}m}{%
    \group_begin:%
        \IfNoValueF{#1}{%
            \keys_set:nn{siunitx}{#1}%
        }%
        \siunitx_hmsP_output:nnn #2%
    \group_end:%
}%
\cs_new_protected:Npn\siunitx_hmsP_output:nnn #1#2#3{%
    \qty[parse-numbers=false]{\mbox{
        \tl_if_blank:nF{#1}{\IfNoValueF{#1}{%
            #1%
            \tl_if_blank:nT{#2}{%
                \tl_if_blank:nF{#3}{\IfNoValueF{#3}{:}}%
            }\tl_if_blank:nF{#2}{\IfNoValueF{#2}{:}}%
        }}%
        \tl_if_blank:nF{#2}{\IfNoValueTF{#2}{%
            \tl_if_blank:nF{#1}{\IfNoValueF{#1}{\tl_if_blank:nF{#3}{\IfNoValueF{#3}{0:}}}}%
        }{%
            #2%
            \tl_if_blank:nF{#3}{\IfNoValueF{#3}{:}}%
        }}\tl_if_blank:nT{#2}{%
            \tl_if_blank:nF{#1}{\IfNoValueF{#1}{\tl_if_blank:nF{#3}{\IfNoValueF{#3}{0:}}}}%
        }%
        \tl_if_blank:nF{#3}{\IfNoValueF{#3}{%
            #3%
        }}%
    }}{%
        \tl_if_blank:nT{#1}{%
            \tl_if_blank:nT{#2}{%
                \tl_if_blank:nF{#3}{\IfNoValueF{#3}{\second}}%
            }\tl_if_blank:nF{#2}{\IfNoValueF{#2}{\minute}}%
        }\tl_if_blank:nF{#1}{\IfNoValueF{#1}{\hour}}%
    }%
}%
% Hour-Minute-Second-Clock (For pasting Time of Day / Clock Time. Like \hmsC{10;20;30} -> "10h 20 min 30s"
\NewDocumentCommand\hmsC{o >{\SplitArgument{2}{;}}m}{%
    \group_begin:%
        \IfNoValueF{#1}{%
            \keys_set:nn{siunitx}{#1}%
        }%
        \siunitx_hmsC_output:nnn #2%
    \group_end:%
}%
\cs_new_protected:Npn\siunitx_hmsC_output:nnn #1#2#3{%
    \IfNoValueF{#1}{%
        \tl_if_blank:nF{#1}{%
            \qty{#1}{\hour}%
            \IfNoValueF{#2}{~}%
        }%
    }%
    \IfNoValueF{#2}{%
        \tl_if_blank:nF{#2}{%
            \qty{#2}{\minute}%
            \IfNoValueF{#3}{~}%
        }%
    }%
    \IfNoValueF{#3}{%
        \tl_if_blank:nF{#3}{%
            \qty{#3}{\second}%
        }%
    }%
}%
\ExplSyntaxOff
% - - - - - - - - - - -
% Fraction Example with this:
%     $2\frac{7}{2}$, \nicefrac{7}{2}. \num[parse-numbers=false]{\frac{7}{2}}, \num[parse-numbers=false]{\nicefrac{1}{3}}.
%     \sfrac{1}{2}, $\sfrac{1}{2}$, $\mathbf{3\times\sfrac{1}{2}}$; \quad \fontfamily{ppl}\selectfont Palatino: \sfrac{1}{2}, \quad \fontfamily{ptm}\selectfont Times: \sfrac{1}{2}
% - - - - - - - -
%
%/\/\/\/\/\/\/\/\/\/\/\/\/\/\/\/\/\/\/\/\/\/\/\/\/\/\/\/\/\/\/\/\/\/\/\/\