Nesting tikzpictures
    [ i.e. create another tikzpicture inside the content/text of a node (\node(){\tikz{...}}) ]
is actually discouraged and thus not guaranteed to work fully as expected.
E.g. there may occur shifts in objects, coordinates are not properly accessible, properties inherit, scaling and other transformations may mess up.
For simple stuff, it may satisfy you, for more complex things, I suggest one of the following.

    | Sidenote:
        [ At least for such stuff, draw/create/compile/render/include your tikzpictures as standalone-docs. That way, the pics are nicely isolated from the actual document. That saves resources, as for instance variables and registers (refer to savebox) are only instantiated inside the tikzpic and vanish afterwards. Also you can utilize functions like "buildnew", so your compilation time greatly improves since pics do not have to be compiled repeatedly as long as they don't change. ]
        [ I have to say: I do pretty much ALWAYS use standalone for tikzpics and then include them. It's pretty neat. ]




#####################
1) Instead of nesting tikzpictures, use a "scope" with "local bounding box"
#####################
\documentclass{article}
\thispagestyle{empty}
\usepackage{tikz}
\usetikzlibrary{fit}
\begin{document}
\begin{figure}
\begin{tikzpicture}
\begin{scope}[local bounding box=inset]
        \coordinate(v) at (3, 4);
        \node(t) at (v){v};
        \draw[->] (1,2) -- (v);
\end{scope}
  \node(t2)[blue] at (v){v};
  \draw[->, red](5, -2) -- (v);

\draw [red] (inset.east) -- (inset.west);
\end{tikzpicture}
\end{figure}
\end{document}




#####################
2) Use saveboxes to then safely nest tikzpictures
    (-> Coordinates will be remembered. The way [remember picture] works is to save the origin location in the aux file, so you have to run the code twice.)
    (It should be noted that by placing the picture inside a node you are centering it (default) at the node location (origin).)
#####################
\documentclass{article}
\thispagestyle{empty}
\usepackage{graphicx}
\usepackage{tikz}
\begin{document}
\begin{figure}
\sbox0{\begin{tikzpicture}[remember picture]
        \coordinate(v) at (3, 4);
        \node(t) at (v){v};
        \draw[->] (1,2) -- (v);
        \end{tikzpicture}
}
  \begin{tikzpicture}[remember picture]
    \node(inset) {\usebox0};
    \node(t2)[blue] at (v){v};
    \draw[->, red](5, -2) -- (v);
  \end{tikzpicture}
\end{figure}
\end{document}




#####################
3) In certain cases, for certain features (accessing coordinates of nested picture) it might work to append the "remember picture" property
    (I consider this rather clumsy)
#####################
\documentclass{article}
\thispagestyle{empty}
\usepackage{tikz}
\begin{document}
  \tikzstyle{every picture}+=[remember picture]
  \begin{tikzpicture}
    \node(inset) {
      \begin{tikzpicture}[remember picture]
       \coordinate(v) at (3, 4);
       \node(t) at (v){v};
       \draw[->] (1,2) -- (v);
      \end{tikzpicture}
    };
    \node(t2)[blue] at (v){v};
    \draw[->, red](5, -2) -- (v);
  \end{tikzpicture}
\end{document}