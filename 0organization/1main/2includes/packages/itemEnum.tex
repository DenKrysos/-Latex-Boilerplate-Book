%________________________________________________________________________
%------------------------------------------------------------------------
%       Itemization & Enumeration (& Description) Lists
%/\/\/\/\/\/\/\/\/\/\/\/\/\/\/\/\/\/\/\/\/\/\/\/\/\/\/\/\/\/\/\/\/\/\/\/\
\let\labelindent\relax%
\usepackage[inline]{enumitem}%
%
%
%
%
%
% \makeatletter\AddToHook{begindocument/before}{%
% }\makeatother%
%________________________________________________________________________
%------------------------------------------------------------------------
%							Setups für enumitem
%					Lists - (itemize, enumerate, description)
%/\/\/\/\/\/\/\/\/\/\/\/\/\/\/\/\/\/\/\/\/\/\/\/\/\/\/\/\/\/\/\/\/\/\/\/\
%  More in the "_perDoc.tex" File inside "./0organization/1main/"
%    -> The "_perDoc.tex" comes after the "_basic.tex" and hence its Settings overwrite.
%
%---------------------------
% %  Additional new List-Style, "Enumeration-Inline-Roman" for enumerating with lower-case roman letters inline
\newlist{enuminlrom}{enumerate*}{1}%
\setlist[enuminlrom]{label=\textit{(\roman*)}}%
% - - - - - - - -
% % Additional List-Style, "Enumeration-Paragraph-Roman" for enumeration with upper-case roman letters, without leftmargin, i.e. looks and behaves like a paragraph, but with enumerated label
\newlist{enumparrom}{enumerate}{1}%
\setlist[enumparrom]{%
    label=\textit{\textbf{(}\Roman*\textbf{)}},%
    %wide,%
    align=left,%
    labelindent=0.5\parindent,%
    listparindent=\parindent,%
    leftmargin=0pt,%
    labelwidth=!,%
    itemindent=!,%
    topsep=0ex,%
    itemsep=0ex,%
    parsep=0ex,%
}%
%
%---------------------------
% %  List-Style Settings
% - - - - - - - -
%\setlist[1]{\labelindent=\parindent} % < Usually a good idea
%\setlist[description]{font=\sffamily\bfseries} % the Default
\setlist[description]{%
    topsep=0ex,%
    itemsep=0ex,%
    parsep=0ex,%
    labelindent=0pt,%
    leftmargin=0pt%
}%
\setlist[itemize]{%
    topsep=0ex,%
    itemsep=0ex,%
    parsep=0ex,%
    leftmargin=*,%1em%
}%
\setlist[enumerate]{%
    topsep=0ex,%
    itemsep=0ex,%
    parsep=0ex,%
    leftmargin=1.5em%
}%
%
\providecommand{\DenKrDescriptionlabelFont}{}%
\providecommand{\DenKrDescriptionlabelFormat}{}%
\providecommand{\DenKrDescriptionlabelMake}{}%
%
\renewcommand{\DenKrDescriptionlabelFont}{\normalfont\bfseries}%
% \renewcommand{\DenKrDescriptionlabelFont}{\normalfont\color{DenKrColor_DescriptionLabel}\bfseries}%
% \renewcommand{\DenKrDescriptionlabelFont}{\normalfont\itshape}%
% \renewcommand{\DenKrDescriptionlabelFont}{\normalfont\color{DenKrColor_DescriptionLabel}\itshape}%
%
\renewcommand{\DenKrDescriptionlabelFormat}[1]{#1:}%
%
\renewcommand{\DenKrDescriptionlabelMake}[1]{\DenKrDescriptionlabelFormat{{\DenKrDescriptionlabelFont#1}}}%
\renewcommand{\descriptionlabel}[1]{\hspace{\labelsep}\DenKrDescriptionlabelMake{#1}}%
%
%
%
%
%
%================================================
% %  Molding some Label-Symbols into Keys
%=================================
% The actual symbols are defined in the "2includes/macros.tex"
%---------------------------
% %  The Default ones
% - - - - - - - -
\SetEnumitemKey{labItDefaultA}{label=\textbullet,align=right,leftmargin=*}%
\SetEnumitemKey{labItDefaultB}{label=\textbf{--},align=right,leftmargin=*}%
\SetEnumitemKey{labItDefaultC}{label=$\mathrm{\ast}$,align=right,leftmargin=*}%
\SetEnumitemKey{labItDefaultD}{label=\textperiodcentered,align=right,leftmargin=*}%
%---------------------------
% %  My "Prägnanz" Symbols, i.e. are supposed to indicate a rising "Conciseness"
%    (Their leftmargins are set, so that they align with each other
% - - - - - - - -
\SetEnumitemKey{labpragA}{label=\labpragAsym,align=right,leftmargin=1.1em}%1.25em
\SetEnumitemKey{labpragB}{label=\labpragBsym,align=right,leftmargin=1.1em}%1.3em
\SetEnumitemKey{labpragC}{label=\labpragCsym,align=right,leftmargin=1.1em}%1.3em
\SetEnumitemKey{labpragD}{label=\labpragDsym,align=right,leftmargin=1.1em}%1.2em
\SetEnumitemKey{labpragE}{label=\labpragEsym,align=right,leftmargin=1.1em}%1.3em
%---------------------------
% %  Simply some more Symbols, suitable as Itemization Labels
% - - - - - - - -
\SetEnumitemKey{labDKBullet}{label=\labDKsymBullet,align=right,leftmargin=*}%1.0em
\SetEnumitemKey{labDKGeviert}{label=\labDKsymGeviert,align=right,leftmargin=*}%1.0em
\SetEnumitemKey{labDKAst}{label=\labDKsymAst,align=right,leftmargin=*}%1.0em
\SetEnumitemKey{labDKDot}{label=\labDKsymDot,align=right,leftmargin=*}%0.9em
\SetEnumitemKey{labDKOblong}{label=\labDKsymOblong,align=right,leftmargin=*}%
\SetEnumitemKey{labDKPointright}{label=\labDKsymPointright,align=right,leftmargin=*}%1.3em
\SetEnumitemKey{labDKTriangle}{label=\labDKsymTriangle,align=right,leftmargin=*}%1.0em
\SetEnumitemKey{labDKTriangleBl}{label=\labDKsymTriangleBl,align=right,leftmargin=*}%1.05em
\SetEnumitemKey{labDKRTriCurvedB}{label=\labDKsymRTriCurvedB,align=right,leftmargin=*}%
\SetEnumitemKey{labDKLozenge}{label=\labDKsymLozenge,align=right,leftmargin=*}%1.0em
\SetEnumitemKey{labDKLozengeBl}{label=\labDKsymLozengeBl,align=right,leftmargin=*}%1.0em
\SetEnumitemKey{labDKSquare}{label=\labDKsymSquare,align=right,leftmargin=*}%1.0em
\SetEnumitemKey{labDKSquareBl}{label=\labDKsymSquareBl,align=right,leftmargin=*}%1.0em
\SetEnumitemKey{labDKDiamondSplit}{label=\labDKsymDiamondSplit,align=right,leftmargin=*}%1.1em
\SetEnumitemKey{labDKCrossA}{label=\labDKsymCrossA,align=right,leftmargin=*}%1.1em
\SetEnumitemKey{labDKCrossB}{label=\labDKsymCrossB,align=right,leftmargin=*}%1.1em
\SetEnumitemKey{labDKCrossC}{label=\labDKsymCrossC,align=right,leftmargin=*}%1.1em
\SetEnumitemKey{labDKCrossD}{label=\labDKsymCrossD,align=right,leftmargin=*}%1.1em
\SetEnumitemKey{labDKCrossE}{label=\labDKsymCrossE,align=right,leftmargin=*}%1.1em
%
%
%
%
%
%================================================
% %  List-Style Further Refining
%=================================
\newlist{itemizeDefault}{itemize}{4}%
\newlist{enumerateDefault}{enumerate}{4}%
\setlistdepth{9}%
%
\renewlist{itemize}{itemize}{9}%
\setlist[itemize,1]{labDKBullet,leftmargin=*}%
\setlist[itemize,2]{labDKOblong,leftmargin=*}%
\setlist[itemize,3]{labDKLozengeBl,leftmargin=*}%
\setlist[itemize,4]{labDKRTriCurvedB,leftmargin=*}%
\setlist[itemize,5]{labDKDiamondSplit,leftmargin=*}%
\setlist[itemize,6]{labDKAst,leftmargin=0.9em}%
\setlist[itemize,7]{labDKSquareBl,leftmargin=*}%
\setlist[itemize,8]{labDKPointright,leftmargin=1.1em}%
\setlist[itemize,9]{labDKDot,leftmargin=*}%
%
\renewlist{enumerate}{enumerate}{9}%
\setlist[enumerate,1]{label=\arabic*.}%
\setlist[enumerate,2]{label=\alph*)}%
\setlist[enumerate,3]{label=\roman*.}%
\setlist[enumerate,4]{label=\Alph*)}%
\setlist[enumerate,5]{label=\Roman*.}%
\setlist[enumerate,6]{label=(\arabic*)}%
\setlist[enumerate,7]{label=(\roman*)}%
\setlist[enumerate,8]{label=(\Roman*)}%
\setlist[enumerate,9]{label=(\alph*)}%
%
%
% - - Templates
%       For Templates, have a look into the corresponding File inside the "8templates" Directory
%/\/\/\/\/\/\/\/\/\/\/\/\/\/\/\/\/\/\/\/\/\/\/\/\/\/\/\/\/\/\/\/\/\/\/\/\
%							enumitem fertig
%------------------------------------------------------------------------
%________________________________________________________________________