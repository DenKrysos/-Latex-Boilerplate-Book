% !TEX root = ../../1main.tex
%
\section{Usage Manual}

This document is optimized for both, being printed as well as digital reading.
It follows typographic guidelines and is appropriately typeset for being printed.
Nonetheless is it equipped with feature, beneficial in digital form:

\begin{itemize}
\item
    The \textit{Page-Number} on the Bottom is a clickable Hyperlink jumping back to the \hyperref[chap:ToC]{Table-of-Contents}.
\item
    The \textit{Headmark}, showing the current Chapter on even (left) pages, respectively the current Section of odd (right) pages is a clickable Hyperlink to the beginning of this Chapter/Section.
\item
    \hyperref[chap:Glossary]{\gerguiquote{\nameref*{chap:Glossary}}}
    \begin{itemize}
    \item
        \textit{Abbreviations}/\textit{Acronyms} in the text are clickable Hyperlinks to the corresponding entry in the \hyperref[sec:Acronyms]{\gerguiquote{\nameref*{sec:Acronyms}}} register in the \nameref{part:appendix}.
    \item
        Defined \textit{Terms} are clickable Hyperlinks to the corresponding entry in the \hyperref[sec:Glossaries]{\gerguiquote{\nameref*{sec:Glossaries}}} register in the \nameref{part:appendix}.
    \item
        Used mathematical/scientific \textit{Symbols} (greek letters, \ldots) are clickable Hyperlinks to the corresponding explaining entry in the \hyperref[sec:Symbols]{\gerguiquote{\nameref*{sec:Symbols}}} register in the \nameref{part:appendix}.
    \item
        They may contain cross-referencing Hyperlinks.
    \end{itemize}
\item
    Register of \hyperref[chap:literature]{\gerguiquote{Bibliography / Literature}}:
    \hyperref[chap:literature]{\gerguiquote{\nameref*{chap:literature}}}
    \begin{itemize}
    \item
        \textit{Literature} References are clickable Hyperlinks leading to the corresponding entry in the \hyperref[chap:literature]{\gerguiquote{Bibliography}} in the \nameref{part:appendix}.
    \item
        \textit{Literature} References are on \textit{first} citation marked with an Asterisk ($\rightarrow$ [$x$]$^\ast$) and printed as a footnote on this page as \enquote{Title | Authors [max 3] | Year}. (If typeset with the correct command of mine.)
        \begin{itemize}
        \item
            This of course additionally to the reference being added an entry to a proper \textit{Literature} Register in the Document's back matter.
        \end{itemize}
    \item
        There may be multiple Literature Registers, like \textit{primary}, \textit{secondary}, a distinct \textit{Technical Specifications} section, etc. To be decided per Document.
    \end{itemize}
\item
    \textit{References} of Elements from within the document (Section, Figure, etc.) are clickable Hyperlinks to this Element.
\item
    On some occasions may be \textit{explicit Hyperlinks} inserted for assisting navigation through the document:
    \begin{itemize}
    \item
        Temporarily in the Footer of a page.
    \item
        Ancillary content registers or single reference links throughout the document.
    \end{itemize}
\end{itemize}

\npi%
\textbf{Sidenote}:
\begin{itemize}
\item
    With most modern pdf-readers, one can smoothly follow a Hyperlink -- e.g. to the Glossary for looking up an acronym -- and use the \enquote{\texttt{Back} feature} (e.g. Mouse Button~4) to comfortably jump back to where the document was scrolled prior to clicking the link -- even across multiple steps.
\end{itemize}