\setkomafont{section}{\sffamily\bfseries\Large\color{DenKrKomaColor_SectionHeading}}%
\RedeclareSectionCommand[%
    beforeskip=7ex,%
    afterskip=2ex,%
]{section}%
\renewcommand*{\sectionformat}{\usekomafont{section}\thesection}%
%
\colorlet{secTFrame}{DenKrKomaColor_ChapterHeading!60!white}%
\colorlet{secTBG}{DenKrKomaColor_ChapterHeading_2!60!white}%
\colorlet{secTBGDark}{secTBG!60!black}%
\colorlet{secTBGFakeEdge}{black!80!white}%
%
\renewcommand*{\DenKrKOMAHookFormatSection}[4]{%
    \def\bandIndentOut{0.7cm}%
    \def\bandIndentIn{0.2cm}%
    \Ifthispageodd{%Right
        \def\secTitleIndent{\bandIndentIn}%
    }{%Left
        \def\secTitleIndent{0pt}%
    }%
    \begin{tikzpicture}[remember picture]%
    \begin{pgfonlayer}{foreground}%
        \coordinate(section_anchor)at(0,0);%
        \node[draw=none,line width=0pt,shape=rectangle,anchor=north west,inner sep=0pt](section_title)at($(section_anchor)+(\secTitleIndent,0)$){%
            \parbox[b]{\dimexpr\textwidth-\bandIndentIn\relax}{%
                #3
                \scalebox{0.8}{\textcolor{DenKrKomaColor_ChapterHeading}{\DenKrKomaHeadingBarockCrossAChar}}
                #4%
            }%
        };%
    \end{pgfonlayer}%
        % Set some values/sizes
        \path let\n1={0.8ex},\n2={5em}in%
            coordinate(bgSepOut)at(\bandIndentOut,\n1)%
            coordinate(bgSepIn)at(\bandIndentIn,\n1)%
            coordinate(bgFadeLen)at(\n2,0.5*\n2)%X: Out, Y: In
            ;%
        % Update bounding-box and thus make the shift of the title node take effect. (Adding also the Y-component creates the appropriate size downwards)
        \path[draw=none,line width=0pt]let\p1=(bgSepOut)in(section_anchor)--($(section_title.south west)-(0,0.5*\y1+0.1ex)$);%
        % "Bake" the previous typeset pieces and "decouple" a new bounding box.
        \useasboundingbox%
            (current bounding box.north west)%
            rectangle%
            (current bounding box.south east);%
        \begin{pgfonlayer}{background}%
        \Ifthispageodd{%Right
            \path let\p1=(bgSepOut),\p2=(bgSepIn),\p3=(bgFadeLen)in%
                coordinate(bgLT)at($(section_title.north west)+(0,\y2)$)%
                coordinate(bgLB)at($(section_title.south west)+(0,-\y2)$)%
                coordinate(bgLC)at($(section_title.west)+(-\x2,0)$)%
                coordinate(bgRT)at($(section_title.north east)+(\x1,\y1)$)%
                coordinate(bgRB)at($(section_title.south east)+(\x1,-\y1)$)%
                coordinate(bgFadeLT)at($(bgLT)+(\y3,0)$)%
                coordinate(bgFadeLB)at($(bgLB)+(\y3,0)$)%
                coordinate(bgFadeRT)at($(bgRT)+(-\x3,0)$)%
                coordinate(bgFadeRB)at($(bgRB)+(-\x3,0)$)%
                coordinate(lappetR_T)at($(bgRB)-(\bandIndentOut-0.8ex,0)$)%
                coordinate(lappetR_B)at($(lappetR_T)+(0,-2ex)$)%
                ;%
            %
            \path[draw=secTFrame,fill=secTBGDark,line width=0.1ex,line join=round](bgRB)--(lappetR_T)--(lappetR_B)--cycle;%
            \path[draw=secTBGFakeEdge,line width=0.15ex](lappetR_T)--(lappetR_B)--++(0,-0.3ex)coordinate(lappetR_Bfadeout);%
            %Why this first unconnected node here: Doesn't do anything for the path. But without it, some inverse-matrix fancy shenanigan shit fails to dimension too large. The coordinate extends the bounding box to the side, nothing else.
            \path[draw=secTBGFakeEdge,line width=0.15ex,path fading=south]($(lappetR_Bfadeout)+(-0.1ex,0)$)(lappetR_Bfadeout)--++(0,-1.2ex);%
            \path[shade,left color=white,right color=secTBG](bgRT)--(bgRB)--(bgFadeRB)--(bgFadeRT)--cycle;%
            \path[shade,left color=white,right color=secTFrame,line cap=round,line join=round]%
                let\n1={0.2ex}in%
                ($(bgRT)+(0,0.1ex)$)--(current subpath start-|bgFadeRT)--++(0,-\n1)-|cycle%
                ($(bgRB)+(0,-0.1ex)$)--(current subpath start-|bgFadeRB)--++(0,\n1)-|cycle%
            ;%
            \path[draw=secTFrame,line width=0.2ex,line cap=round,line join=round]%
                (bgRT)--(bgRB);%
            %
            %
            \path[shade,left color=secTBG,right color=white]let\n1={40}in%
                (bgLT)to[out=180+\n1,in=90](bgLC)to[in=180-\n1,out=270](bgLB)%
                --(bgFadeLB)--(bgFadeLT)--cycle;%
            \path[shade,left color=secTFrame,right color=white,line cap=round,line join=round]%
                let\n1={0.2ex}in%
                ($(bgLT)+(0,0.1ex)$)--(current subpath start-|bgFadeLT)--++(0,-\n1)-|cycle%
                ($(bgLB)+(0,-0.1ex)$)--(current subpath start-|bgFadeLB)--++(0,\n1)-|cycle%
            ;%
            \path[draw=secTFrame,line width=0.2ex,line cap=round,line join=round]let\n1={40}in%
                (bgLT)to[out=180+\n1,in=90](bgLC)to[in=180-\n1,out=270](bgLB)%
                ;%
        }{%Left
            \path let\p1=(bgSepOut),\p2=(bgSepIn),\p3=(bgFadeLen)in%
                coordinate(bgLT)at($(section_title.north west)+(-\x1,\y1)$)%
                coordinate(bgLB)at($(section_title.south west)+(-\x1,-\y1)$)%
                coordinate(bgRT)at($(section_title.north east)+(0,\y2)$)%
                coordinate(bgRB)at($(section_title.south east)+(0,-\y2)$)%
                coordinate(bgRC)at($(section_title.east)+(\x2,0)$)%
                coordinate(bgFadeLT)at($(bgLT)+(\x3,0)$)%
                coordinate(bgFadeLB)at($(bgLB)+(\x3,0)$)%
                coordinate(bgFadeRT)at($(bgRT)+(-\y3,0)$)%
                coordinate(bgFadeRB)at($(bgRB)+(-\y3,0)$)%
                coordinate(lappetL_T)at($(bgLB)+(\bandIndentOut-0.8ex,0)$)%
                coordinate(lappetL_B)at($(lappetL_T)+(0,-2ex)$)%
                ;%
            %
            \path[draw=secTFrame,fill=secTBGDark,line width=0.1ex,line join=round](bgLB)--(lappetL_T)--(lappetL_B)--cycle;%
            \path[draw=secTBGFakeEdge,line width=0.15ex](lappetL_T)--(lappetL_B)--++(0,-0.3ex)coordinate(lappetL_Bfadeout);%
            %This first unconnected Node: See above
            \path[draw=secTBGFakeEdge,line width=0.15ex,path fading=south]($(lappetL_Bfadeout)+(-0.1ex,0)$)(lappetL_Bfadeout)--++(0,-1.2ex);%
            \path[shade,left color=secTBG,right color=white](bgLT)--(bgLB)--(bgFadeLB)--(bgFadeLT)--cycle;%
            \path[shade,left color=secTFrame,right color=white,line cap=round,line join=round]%
                let\n1={0.2ex}in%
                ($(bgLT)+(0,0.1ex)$)--(current subpath start-|bgFadeLT)--++(0,-\n1)-|cycle%
                ($(bgLB)+(0,-0.1ex)$)--(current subpath start-|bgFadeLB)--++(0,\n1)-|cycle%
            ;%
            \path[draw=secTFrame,line width=0.2ex,line cap=round,line join=round]%
                (bgLT)--(bgLB);%
            %
            %
            \path[shade,left color=white,right color=secTBG]let\n1={40}in%
                (bgRT)to[out=0-\n1,in=90](bgRC)to[in=0+\n1,out=270](bgRB)%
                --(bgFadeRB)--(bgFadeRT)--cycle%
                ;%
            \path[shade,left color=white,right color=secTFrame,line cap=round,line join=round]%
                let\n1={0.2ex}in%
                ($(bgRT)+(0,0.1ex)$)--(current subpath start-|bgFadeRT)--++(0,-\n1)-|cycle%
                ($(bgRB)+(0,-0.1ex)$)--(current subpath start-|bgFadeRB)--++(0,\n1)-|cycle%
            ;%
            \path[draw=secTFrame,line width=0.2ex,line cap=round,line join=round]let\n1={40}in%
                (bgRT)to[out=0-\n1,in=90](bgRC)to[in=0+\n1,out=270](bgRB)%
                ;%
        }%
    \end{pgfonlayer}%
    \end{tikzpicture}%
    % \par%
}%