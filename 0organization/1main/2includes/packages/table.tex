\usepackage{array}
%
%%%%\usepackage{tabular}% Latex-Std, just don't use
%\usepackage{tabular*}% Latex-Std, bit better
%%%%\usepackage{tabularx}% More comfortable than, 'tabular*', better to use, but no nesting
\usepackage{tabulary}% Nearly the same as 'tabularx', bit more extensive
\usepackage{makecell}% http://ctan.org/pkg/makecell
\usepackage{diagbox}% package "makecell" also has a command \diaghead{}{}{}
%------------------------------------------------------------------------------------
% -- Table recommendation:
% ---- Use 'tabulary', together with 'makecell' (if necessary)
% ---- -- Could also use 'tabularx', together with 'makecell', if you like it better...
% ---- Otherwise maybe 'tabular*'
%------------------------------------------------------------------------------------
%longtable is a special one, which allows page-breaks
\usepackage{longtable}%
%----------------------------------------------------------------
%
\usepackage{multirow}% Create tabular cells spanning multiple rows
\usepackage{colortbl}% Add colour to LaTeX tables
\usepackage{booktabs}%
\usepackage{hhline}%
%
%----------------------------------------------------------------
% A mixed package. Serves for Tabulars, Arrays and Matrices
\usepackage{nicematrix}%