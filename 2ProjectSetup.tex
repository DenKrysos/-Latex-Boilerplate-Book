% !TEX root = 1main.tex
%==================================================================================
% ----  Project-Settings
%----------------------------------------------------------------------------------
%
%======================
%== Valid Layout Values
% - scrbook % Optimized "book" for Using KOMA-Script.
% - scrbook_print % Pretty much the same as "scrbook" but optimized for a print-Version in that it uses a different Page-Geometry and perhaps font-size
%  %  %  %  %
% - scrartcl % The KOMA variant for "Article". For smaller, more concise documents. Has no Chapter. E.g. for "scientific Papers". But in this project optimized for using KOMA-Script
%  %  %  %  %
% - tikz_standalone % Sort of a special layout type. Used for Tikz-Pictures compiled in standalone-mode
% = = = Not for Paper but big Documents, like books = = = = =
% - scrbook
\newcommand{\DenKrLayout}{scrbook}%
\newcommand{\DenKrLayoutLanguage}{english}% en_ger, english, ngerman
\newcommand{\DenKrLayoutUseHyperref}{1}% 0: false aka NOT use Hyperref  |  1: true aka USE Hyperref
%
% _ _ _ _ _ _ _ _ _ _ _ _ _ _ _ _ _ _ _ _ _ _ _ _ _ _ _ _ _ _ _ _
% --  Some Stuff to help during working state
% Enables you this infamous comments that helps gratuitously while concurrent working / reviewing among several Co-Authors.
% Have a look into the File  {"\DenKrSupplyRootDir/DenKr_comments".tex}  to adjust Author-&-Macro-Names to the needs of your project
% During writing, set the following to '1' to have the Comment-Macros be printed.
% After finishing the work, you can just set it to '0' to make every Comment-Output disappear.
% Additional Feature: The Macro \disablewr{}. Its Argument is eaten away / vanishes / has no effect when \DenKrCommentsUsage is set to '0'
\newcommand{\DenKrCommentsUsage}{1}% 1: Enabled  |  0: Disabled, Comment-Macros don't do anything
%----------------------------------------------------------------------------------
% - - - - - - - - - - - - - - - - - - - - - - - - - - - - - - - - - - - - - - - - -
%==================================================================================
%
%
%
%
%==================================================================================
% ----  Project-Setup
% ----     (In most cases, you shouldn't be required to touch anything below)
%----------------------------------------------------------------------------------
\providecommand{\DenKr}{}% Primarily used for checking whether it is executed within a "DenKr" Environment (standalone, ...)
%
%Consider setting bevor \input{}ing this file: (Setting with any Directory)
%      \newcommand{\DenKrSubDirPrefix}{./}%
\providecommand{\DenKrSubDirPrefix}{}%
%----------------------------------------------
%----------------------------------------------
\newcommand{\DenKrOrgaRootDirPATH}{0organization}%
\newcommand{\DenKrLayoutMainRootDirPATH}{\DenKrOrgaRootDirPATH/1main}%
\newcommand{\DenKrLayoutBaseRootDirPATH}{\DenKrOrgaRootDirPATH/2layout}%
\newcommand{\DenKrLayoutRootDirPATH}{\DenKrLayoutBaseRootDirPATH/\DenKrLayout}%
\newcommand{\DenKrSupplyRootDirPATH}{1supply}%
\newcommand{\DenKrContentRootDirPATH}{9chapter}%
\newcommand{\DenKrTablesRootDirPATH}{5tables}%
\newcommand{\DenKrListingsRootDirPATH}{6listings}%
\newcommand{\DenKrTikzRootDirPATH}{7Tikz}%
\newcommand{\DenKrGraphicsRootDirPATH}{8graphics}%
\newcommand{\DenKrAlgorithmRootDirPATH}{\DenKrListingsRootDirPATH}%
\newcommand{\DenKrLiteratureDirPATH}{\DenKrSupplyRootDirPATH}%
% - - - - - - - - - - - - - - - - - - - -
\newcommand{\DenKrOrgaRootDir}{\DenKrSubDirPrefix\DenKrOrgaRootDirPATH}%
\newcommand{\DenKrLayoutMainRootDir}{\DenKrSubDirPrefix\DenKrLayoutMainRootDirPATH}%
\newcommand{\DenKrLayoutBaseRootDir}{\DenKrSubDirPrefix\DenKrLayoutBaseRootDirPATH}%
\newcommand{\DenKrLayoutRootDir}{\DenKrSubDirPrefix\DenKrLayoutRootDirPATH}%
\newcommand{\DenKrSupplyRootDir}{\DenKrSubDirPrefix\DenKrSupplyRootDirPATH}%
\newcommand{\DenKrContentRootDir}{\DenKrSubDirPrefix\DenKrContentRootDirPATH}%
\newcommand{\DenKrTablesRootDir}{\DenKrSubDirPrefix\DenKrTablesRootDirPATH}%
\newcommand{\DenKrListingsRootDir}{\DenKrSubDirPrefix\DenKrListingsRootDirPATH}%
\newcommand{\DenKrTikzRootDir}{\DenKrSubDirPrefix\DenKrTikzRootDirPATH}%
\newcommand{\DenKrGraphicsRootDir}{\DenKrSubDirPrefix\DenKrGraphicsRootDirPATH}%
\newcommand{\DenKrAlgorithmRootDir}{\DenKrSubDirPrefix\DenKrAlgorithmRootDirPATH}%
\newcommand{\DenKrLiteratureDir}{\DenKrSubDirPrefix\DenKrLiteratureDirPATH}%
%----------------------------------------------
%----------------------------------------------
\newcommand{\DenKrSegmentationSubDirPATH}{\DenKrContentRootDirPATH/0segmentation}%
\newcommand{\DenKrSegmentationSubDir}{\DenKrSubDirPrefix\DenKrSegmentationSubDirPATH}%
%----------------------------------------------
%----------------------------------------------
\newcommand{\DenKrTikzArtDirPATH}{\DenKrLayoutMainRootDirPATH/8templates/tikz/7Tikz}%
\newcommand{\DenKrTikzArtDir}{\DenKrSubDirPrefix\DenKrTikzArtDirPATH}%
%=========================================================================
% ----  Project-Setup (Some little additional for further Structuring)
%-------------------------------------------------------------------------
\newcommand{\DenKrLayoutIncludeBiographies}{1}% 1: Print the Biographies after Literature-References  |  0: Don't print Biographies
%
%
%
%==================================================================================
% ----  More on LaTeX
%----------------------------------------------------------------------------------
\newcommand{\DenKrCompiler}{LuaLaTeX}% LuaLaTeX, pdfLaTeX
%
% - Just saying: Activating Japanese-Fonts (or any CJK) introduces some hickups: Increase compilation/loading time. Also: It patches the "listings.sty" package and while doing so introduces a minor bug (as of 2023-12): They forgot a line-end percentage, which adds an additional space before each \lstinline.
\newcommand{\DenKrJPFont}{0}% 1: Also make Japanese Fonts available  |  0: Don't setup Japanese Fonts
%
% correct bad hyphenation here
\hyphenation{%
	op-ti-cal
    net-works
    semi-con-duc-tor
    time-stamp
}%
%
%
%
%
%
%
%==================================================================================
% ----  Some 'further Definitions' that shouldn't be required to touch.
%----------------------------------------------------------------------------------
\newcommand{\DenKrLayoutCommonDirPATH}{1common}%
% - - - - - - - - - - - - - - - - - - - -
\newcommand{\DenKrLayoutCommonDir}{\DenKrSubDirPrefix\DenKrLayoutBaseRootDirPATH/\DenKrLayoutCommonDirPATH}%