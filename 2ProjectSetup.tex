%==================================================================================
% ----  Project-Settings
%----------------------------------------------------------------------------------
%
%Valid Layout Values
% - scrbook % Optimized "book" for Using KOMA-Script.
% - scrbook_print % Pretty much the same as "scrbook" but optimized for a print-Version in that it uses a different Page-Geometry and perhaps font-size
%  %  %  %  %
% - tikz_standalone % Sort of a special layout type. Used for Tikz-Pictures compiled in standalone-mode
% = = = Not for Paper but big Documents, like books = = = = =
% - scrbook
\newcommand{\DenKrLayout}{scrbook}%
\newcommand{\DenKrLayoutLanguage}{english}% en_ger, english, ngerman
\newcommand{\DenKrLayoutUseHyperref}{1}% 0: false aka NOT use Hyperref  |  1: true aka USE Hyperref
%
% _ _ _ _ _ _ _ _ _ _ _ _ _ _ _ _ _ _ _ _ _ _ _ _ _ _ _ _ _ _ _ _
% --  Some Stuff to help during working state
% Enables you this infamous comments that helps gratuitously while concurrent working / reviewing among several Co-Authors.
% Have a look into the File  {"\DenKrSupplyRootDir/DenKr_comments".tex}  to adjust Author-&-Macro-Names to the needs of your project
% During writing, set the following to '1' to have the Comment-Macros be printed.
% After finishing the work, you can just set it to '0' to make every Comment-Output disappear.
% Additional Feature: The Macro \disablewr{}. Its Argument is eaten away / vanishes / has no effect when \DenKrCommentsUsage is set to '0'
\newcommand{\DenKrCommentsUsage}{1}% 1: Enabled  |  0: Disabled, Comment-Macros don't do anything
%----------------------------------------------------------------------------------
% - - - - - - - - - - - - - - - - - - - - - - - - - - - - - - - - - - - - - - - - -
%==================================================================================
%
%
%
%
%==================================================================================
% ----  Project-Setup
% ----     (In most cases, you shouldn't be required to touch anything below)
%----------------------------------------------------------------------------------
%
\newcommand{\DenKrOrgaRootDir}{./0organization}%
\newcommand{\DenKrLayoutMainRootDir}{\DenKrOrgaRootDir/1main}%
\newcommand{\DenKrLayoutBaseRootDir}{\DenKrOrgaRootDir/2layout}%
\newcommand{\DenKrLayoutRootDir}{\DenKrLayoutBaseRootDir/\DenKrLayout}%
\newcommand{\DenKrSupplyRootDir}{./1supply}%
\newcommand{\DenKrContentRootDir}{./9chapter}%
\newcommand{\DenKrTablesRootDir}{./5tables}%
\newcommand{\DenKrListingsRootDir}{./6listings}%
\newcommand{\DenKrTikzRootDir}{./7Tikz}%
\newcommand{\DenKrGraphicsRootDir}{./8graphics}%
\newcommand{\DenKrAlgorithmRootDir}{\DenKrListingsRootDir}%
\newcommand{\DenKrLiteratureDir}{\DenKrSupplyRootDir}%
% - - - - - - - - - - - - - - -
\newcommand{\DenKrTikzArtDir}{\DenKrLayoutMainRootDir/8templates/tikz/7Tikz}%
%=========================================================================
% ----  Project-Setup (Some little additional for further Structuring)
%-------------------------------------------------------------------------
\newcommand{\DenKrSegmentationSubDir}{\DenKrContentRootDir/0segmentation}%
\newcommand{\DenKrLayoutIncludeBiographies}{1}% 1: Print the Biographies after Literature-References  |  0: Don't print Biographies
%
%
%
%==================================================================================
% ----  More on LaTeX
%----------------------------------------------------------------------------------
\newcommand{\DenKrCompiler}{LuaLaTeX}% LuaLaTeX, pdfLaTeX
% correct bad hyphenation here
\hyphenation{%
	op-ti-cal
    net-works
    semi-con-duc-tor
    time-stamp
}%