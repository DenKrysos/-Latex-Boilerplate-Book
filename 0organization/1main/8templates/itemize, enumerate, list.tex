%
% \begin{itemize}[label=$\rightarrow$,leftmargin=1.5em,topsep=0ex,itemsep=0ex,parsep=0ex]
%
%
%
%
%
% % \begin{enumerate}[label=\textit{(\roman*)},leftmargin=\widthof{\textit{(iii)}\ }]
{\setcounter{enumi}{0}
\renewcommand{\descriptionlabel}[1]{\hspace{\labelsep}\normalfont\textit{(\stepcounter{enumi}\roman{enumi}) #1}:}
\begin{description}
\item[Text-In-Label-1]%
    Actual Items Text.
\item[Text-In-Label-2]%
    Actual Items Text.
% \end{enumerate}
\end{description}}
%
%
%
% % % %   Inline-Enumeration  (When including the package with the option "inline": \usepackage[inline]{enumitem}, additional environments are enabled. This print the List inline, when adding an asterisk after the list-type.)
\begin{enumerate*}[label=\textit{(\roman*)}]
\item%
    a
\item%
    b
\end{enumerate*}
%
%
% % %  Additionally, you may just use my defined List-Style "enuminlrom" (Enumeration-Inline-Roman)
%
%
%
%
%
% % % %   Indentation in List only in First Line ("Hängender Einzug"). Useful for enumerating big blocks of text
% Either
\setlist[enumerate]{itemindent=\dimexpr\labelwidth+\labelsep\relax,leftmargin=0pt}
Or alternatively
\setlist[enumerate]{wide=\parindent}
%
% % %  You can apply this only to a specific list level:
\setlist[enumerate,1]{wide=\parindent}
%
% % %  And say, you do want to move the second level further to the right after all, add something like
\setlist[enumerate,2]{leftmargin=6em}
%
%
%
%
%
%
\setlist[description]{leftmargin=0em,labelindent=1em,style=unboxed}%
\renewcommand{\descriptionlabel}[1]{\hspace{\labelsep}\normalfont\textbf{#1}:}%