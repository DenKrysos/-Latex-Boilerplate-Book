% !TEX root = ../1main.tex
%

\chapter[Alternative \enquote{Some Chap} Title. Shorter for page-heading]{Some Chap: An absurd Title, way too long to be a reasonable choice, but to see the effect on multi-line}
\label{chap:chapter1}


Example-Citation:
\cite{DenKr_denkrement1_indeco}
\nl%
A custom citation command that puts information about the reference to the footnote on first occurrence:
\citeff{DenKr_MigArb,DenKr_denkrement1_indeco}%

\npi%
Some special characters:
»«.
\nl%
Some Example Macro:
\gerguiquote{My Enquote}.
\nl%
Example Acronym, Glossary-Entry \& Symbol:\nl
\gls{fsa}, \glsunset{api}\gls{api}, \gls{shmem}\nl
\gls{molmass}, \gls{sigma}\nl
\gls{eventfd}

\np
\newcommand{\I}{\mathrm{i}}
$y = \int_0^x\cos(x)\,\mathrm{d}{x} = \frac{e^{\I x} - e^{-\I x}}{2\I} | 0123456789$
\nl
\begin{equation}
y = \int_0^x\cos(x)\,\mathrm{d}{x} = \frac{e^{\I x} - e^{-\I x}}{2\I} | 0123456789
\end{equation}







\section{Font Settings}

Ordinary | \textrm{Serif} | \textsf{Sans-Serif} | \texttt{Monospaced-TrueType} | \textbf{Bold} | \textit{Italic/Kursiv} | \textit{Slanted/Schräggestelt}\nl
| \textsc{SmallCaps (SC)}\nl%
\vspace{0.5\baselineskip}
\newcommand\fontTestText{Xyz Qu Text | 0 l \enquote{enq}}
\begin{tabular}{l|l|l|l}%
\diagbox{\small shape}{\small family}&Ordinary (Serif)&Sans-Serif&Mono-spaced\\%
\hline%
Plain&
    \rmfamily
    \fontTestText
    &
    \sffamily
    \fontTestText
    &
    \ttfamily
    \fontTestText
    \\%
Bold&
    \rmfamily
    \bfseries
    \fontTestText
    &
    \sffamily
    \bfseries
    \fontTestText
    &
    \ttfamily
    \bfseries
    \fontTestText
    \\%
Italic&
    \rmfamily
    \itshape
    \fontTestText
    &
    \sffamily
    \itshape
    \fontTestText
    &
    \ttfamily
    \itshape
    \fontTestText
    \\%
Bold-Italic&
    \rmfamily
    \bfseries
    \itshape
    \fontTestText
    &
    \sffamily
    \bfseries
    \itshape
    \fontTestText
    &
    \ttfamily
    \bfseries
    \itshape
    \fontTestText
    \\%
Slanted&
    \rmfamily
    \slshape
    \fontTestText
    &
    \sffamily
    \slshape
    \fontTestText
    &
    \ttfamily
    \slshape
    \fontTestText
    \\%
\hline%
SmallCaps&
    \rmfamily
    \scshape
    \fontTestText
    &
    \sffamily
    \scshape
    \fontTestText
    &
    \ttfamily
    \scshape
    \fontTestText
    \\%
SC-Italic&
    \rmfamily
    \itshape
    \scshape
    \fontTestText
    &
    \sffamily
    \itshape
    \scshape
    \fontTestText
    &
    \ttfamily
    \itshape
    \scshape
    \fontTestText
    \\%
SC-Bold&
    \rmfamily
    \bfseries
    \scshape
    \fontTestText
    &
    \sffamily
    \bfseries
    \scshape
    \fontTestText
    &
    \ttfamily
    \bfseries
    \scshape
    \fontTestText
    \\%
\end{tabular}%






\np
%
\lstinputlisting[
    frame=single,
    label=lst:waitingChain,
    caption={Example-Listing (for Programming-Language \textit{C})},
    captionpos=b,
    language=DenKr-C,
    morekeywords={[3]{
        some_struct
    }},
    morekeywords={[4]{
        entry,
        member,
        iterator
    }}
]
{\DenKrListingsRootDir/example_lst_C.c}
%



\section{Testing Stuff}
\label{sec:testing}

Document-within References to check whether hyperref \& nameref work properly:\nl%
\ref{chap:intro} (\nameref{chap:intro}). \ref{chap:chapter1}  (\nameref{chap:chapter1}). \ref{sec:testing} (\nameref{sec:testing})\nl%
Consider using my Macros:\nl
\eleref{fig:tikzexample} or \elenumnamref{fig:tikzexample}


\begin{figure*}[!htpb]
\centering
    \includegraphics[width=2em]{{"\DenKrGraphicsRootDir/Quatsch/thumbsup"}.png}
    \caption{A Caption with long enough text to cause a line wrap, for the goal of testing whether the setting puts a hanging indent, which separates the text from the Figure Label.}%
    \label{fig:testFig}
\end{figure*}



\subsection{Enumeration}

\begin{itemize}
\item%
    Example Itemization
\end{itemize}
\begin{enumerate}
\item%
    Example Enumeration
\end{enumerate}
\begin{description}
\item[Description Item Label]%
    An Example Description List, with a longer Parapgraph. Just some random Text without any sense, but with sufficient length to cause a line-break, so that the indentation is properly showing influence.
\end{description}
\begin{enuminlrom}
\item%
    Example Inline Roman Enumeration
\item%
    Item 2.
\end{enuminlrom}


\subsubsection{A-SubSubSection}
With some arbitrary Text.

\paragraph{A-Paragraph}
With some arbitrary Text.

\subparagraph{A-SubParagraph}
With some arbitrary Text.





% \section{\textbackslash gls{} in Heading: \glstextHead{dll}, \glsdescHead{dll}, \glstextplHead{dll}, \glsdescplHead{dll}. And some more words, to make it multi-line to check the configured section-format}
\section{\textbackslash gls\{\} in Heading:
    \texorpdfstring{\glstextHead{dll}}{DLL},
    \texorpdfstring{\glsdescHead{dll}}{DLL},
    \texorpdfstring{\glstextplHead{dll}}{DLLs},
    \texorpdfstring{\glsdescplHead{dll}}{DLLs}.
    And some more words, to make it multi-line to check the configured section-format
}

Just some text.



\edef\tokenA{\glsentrytext{dll}}
\edef\tokenB{\glsentrydesc{dll}}
% \subsection{Alternative Solution: \texorpdfstring{\tokenA}{\tokenA}, \texorpdfstring{\tokenB}{\tokenB}}
\subsection{Alternative Solution: \texorpdfstring{\tokenA}{DLL}, \texorpdfstring{\tokenB}{DLL}}
Hem, in TexLive 2023, something with the Macro Expansion changed, so that is not properly working anymore. Well, haven't investigated on it yet. My apologies.

These macros allow to use glossaries entries in Section-Heading, without the Warning \enquote{Token not allowed}.


\subsection{My Recommendation: 
\glsfmtshort{dll}, \glsfmtshortpl{dll},
\glsfmtlong{dll}, \glsfmtlongpl{dll}
}

Using \enquote{glossaries-extra}, one has access to these Commands above.\nl
So, basically, when using \enquote{glossaries-extra} (as opposed to plain glossaries, without -extra), don't employ my macros from above, but use these \textbackslash glsfmt[\ldots], like in the heading here.


\subsubsection{Some useful Glossaries-Cmds as Reference}
\glsreset{dll}\glspl{dll}\nl
\glstext{dll}, \glsdesc{dll}, \glsplural{dll},\nl
$\rightarrow$ Just saying, the \textbackslash glsdescpl\{\} cmd, used here, is not from the Glossaries package, but defined by DenKr: \glsdescpl{dll}%\glsdescplural{dll}
\nl
\ \ \ (Because the \textbackslash glsdescplural\{\} seems to be always buggy as hell\ldots)

\npi
\glsentryshortpl{dll}, \glsentrylongpl{dll}, \glsentryshort{dll}, \glsentrylong{dll};\nl
\glsentrytext{dll}, \glsentryplural{dll}, \glsentrydesc{dll}, \glsentrydescplural{dll}.\nl
\glsentryshortpl{eventfd}, \glsentrylongpl{eventfd}, \glsentryshort{eventfd}, \glsentrylong{eventfd};\nl
\glsentrytext{eventfd}, \glsentryplural{eventfd}, (Careful with the \enquote{descs} on glossary-entries \emoji{winking-face}. Outcommented here:)%\glsentrydesc{eventfd}, \glsentrydescplural{eventfd}.

\npi
\glsfmtshort{dll}, \glsfmtlong{dll}, \glsfmtshortpl{dll}, \glsfmtlongpl{dll},\nl
\glsfmttext{dll}, \glsfmtname{dll}.
\nl
\glsfmtshort{eventfd}, \glsfmtlong{eventfd}, \glsfmtshortpl{eventfd}, \glsfmtlongpl{eventfd},\nl
\glsfmttext{eventfd}, \glsfmtname{eventfd}.\nl
\ \ \ (As you see above, the fmtshort/long are not available for Glossary-Entries, but for Acronyms. So, in essence, just use \textbackslash glsfmttext\{\} for basically everything, except when you need the long-form of an Acronym.)





\ifdef{\DenKrJPFont}{\ifnumcomp{\number\numexpr\DenKrJPFont\relax}{>}{0}{%
\section{Japanese Font -- 日本語}

Ordinary | \textrm{Serif} | \textsf{Sans-Serif} | \texttt{Monospaced-TrueType} | \textbf{Bold} | \textit{Italic} | \textit{\textbf{BoldItalic}} | \textsc{SmallCaps}\nl%
Some Japanese Characters:\nl%
\begin{tabular}{llll}%
Ordinary:&日本語行無&いきます&チョット\\%
Bold:&\textbf{日本語行無}&\textbf{いきます}&\textbf{チョット}\\%
Serif (Mincho):&\textrm{日本語行無}&\textrm{いきます}&\textrm{チョット}\\%
Sans-Serif (Gothic):&\textsf{日本語行無}&\textsf{いきます}&\textsf{チョット}\\%
Bold-Sans:&\textbf{\textsf{日本語行無}}&\textbf{\textsf{いきます}}&\textbf{\textsf{チョット}}\\%
Monospaced:&\texttt{日本語行無}&\texttt{いきます}&\texttt{チョット}\\%
Mono-Bold:&\textbf{\texttt{日本語行無}}&\textbf{\texttt{いきます}}&\textbf{\texttt{チョット}}\\%
\hline%
Italic:&\textit{日本語行無}&\textit{いきます}&\textit{チョット}\\%
Serif-It:&\textit{\textrm{日本語行無}}&\textit{\textrm{いきます}}&\textit{\textrm{チョット}}\\%
Sans-It:&\textit{\textsf{日本語行無}}&\textit{\textsf{いきます}}&\textit{\textsf{チョット}}\\%
Mono-It:&\textit{\texttt{日本語行無}}&\textit{\texttt{いきます}}&\textit{\texttt{チョット}}\\%
\hline%
SmallCaps:&\textsc{日本語行無}&\textsc{いきます}&\textsc{チョット}\\%
% Serif-SC:&\textsc{\textrm{日本語行無}}&\textsc{\textrm{いきます}}&\textsc{\textrm{チョット}}\\%
% Sans-SC:&\textsc{\textsf{日本語行無}}&\textsc{\textsf{いきます}}&\textsc{\textsf{チョット}}\\%
% Mono-SC:&\textsc{\texttt{日本語行無}}&\textsc{\texttt{いきます}}&\textsc{\texttt{チョット}}\\%
\end{tabular}%
}{%
    \section{Japanese Font}
    Not activated per default, but Boilerplate is ready for it. You can activate \textbackslash DenKrJPFont in \enquote{./2ProjectSetup.tex} to enable them.
}}{}%


\np
Imported Standalone Snippet: -->
\import{\DenKrContentRootDir/}{3020_StandaloneConstituent_Snippet}
<--
% \np
% As Standalone-Inclusion: -->
% \includestandalone[mode=\includestandalonedefaultmode]{\DenKrContentRootDir/3020_StandaloneConstituent_Snippet}%
% <--
% \np


\section{Text with Outline / Contour}
Two different Packages and by that Methods are supplied here:
\begin{itemize}
    \item contour: Prints the actual text as normal and surrounds it with the Contour.
    \item pdfrender: Prints the outline as the actual text and fills this inside with a different Color.
\end{itemize}

\begin{itemize}[labpragA]
    \item contour grows bigger to the outside and makes the Outline sleeked (when growing too big)
    \item pdfrender keeps the outline sharp and well defined, but may let the inside look wonky
\end{itemize}

\begin{itemize}[labpragB]
    \item Methods to be ideally employed depends on the use-case
\end{itemize}

\subsection{Package \enquote{pdfrender}}
{%
    \Huge%
    \definecolor{CharFillColor}{rgb}{1,.8,.8}%
    \colorlet{CharFillColor}{orange}%
    \sffamily%
    \textcolor{violet}{%
    \textpdfrender{%
        TextRenderingMode=FillStroke,
        LineWidth=\DenKrOutlineWidth,
        FillColor=CharFillColor,
        MiterLimit=1,
    }{Text with Outline/Contour}}\nl%
    \textcolor{violet}{%
    \textpdfrender{%
        TextRenderingMode=FillStroke,
        LineWidth=0.06em,
        FillColor=CharFillColor,
        MiterLimit=1,
    }{Text with Outline/Contour}}%
}%

\subsection{Package \enquote{contour}}
{%
    \LARGE%
    \contourlength{\DenKrOutlineWidth}% The Default-Value here
    \contour{violet}{\textcolor{orange}{Text with Outline/Contour}}\nl%
    {%
        \contourlength{0.2em}%
        \contour{violet}{\textcolor{orange}{Text with Outline/Contour}}\nl%
        \contour{violet}{\textcolor{orange}{Text with Outline/Contour}}\nl%
    }%
    \contour{violet}{\textcolor{orange}{Text with Outline/Contour}}\nl%
}%


\np
%
\begin{figure}[!htpb]
    \centering
	\includestandalone[mode=\includestandalonedefaultmode]{\tikzFilesPath/tikztest}% width=\columnwidth
    \caption{Tikz-Picture Caption. Some Example Standalone-Tikz-Pic (Another Example is below outcommented)}
    \label{fig:tikz_test}
\end{figure}
%
%
% \tikzabb%[tex]%
% {tikztest}%
% [\DenKrLayoutMainRootDir/8templates/tikz/7Tikz]% Alternative Path to the std "./7Tikz"
% {%
% Tikz-Picture Caption. Some Example Standalone-Tikz-Pic
% }%
% {1}[1,1,!ht]%[0.55,1,!ht]%
% {fig:tikz_test2}%
%
%



\setcounter{chapter}{9}
\chapter{A Chapter with two-digit number}
\section{And more Sections}
\subsection{to make the ToC two pages}
\section{More Sec}
\section{Yet more Sec}
\subsection{And a Subsec}





\headingOpenTempAny%
\chapter{Chapter opened Left}
\headingOpenTempRestore%

Temporary set to open Chapters on any side, which makes this on the left side.